\chapter*{Introdução}
% Resumo
Neste relatório, conforme previsto pelo projeto proposto, fez-se primeiramente uma revisão de Relatividade Especial, com referência em \cite{dray2012geometry}. Após isso, começou-se o estudo da Teoria da Relatividade Geral, passando pelos tópicos de campos tensoriais, derivadas e geodésicas no espaço curvo. Definem-se, também, os tensores de Riemann, Ricci e de métrica. No fim, chega-se na dedução da métrica de Schwarszchild. A referência principal utilizada foi \cite{foster2010short}, com as referencias complementares \cite{hartle2003gravity} e \cite{schutz2009first}. 

As referências retiradas sobre resultados de eletromagnetismo e mecânica clássica foram retirados, respectivamente, de \cite{griffiths2005introduction} e \cite{nussenzveig2013curso}.

% Atividades realizadas em paralelo
As disciplinas cursadas pelo beneficiário ao longo do período de desenvolvimento deste relatório e suas respectivas notas finais, em parênteses, foram: Controle e Servomecanismo (8,8); Eletromagnetismo 2 (9,7); Engenharia de Dispositivos e Materiais Avançados (9,5); Engenharia Eletroquímica (7,6); Estado Sólido 1 (8,2); Sociologia Industrial e do Trabalho (7,8); Teoria da Informação: Clássica e Quântica (9,3); Teoria das Organizações (7,5).

Todas as figuras utilizadas neste relatório foram feitas pelo autor.


%Resumo do plano inicial
%Resumo do que foi realizado no período a que se refere o relatório
%Plano de trabalho e cronograma para as etapas seguintes, quando houver