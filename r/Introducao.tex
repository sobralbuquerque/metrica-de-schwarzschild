\chapter*{Introdução}
\section*{Atividades realizadas pelo beneficiário ao longo do período de elaboração do relatório}
% Resumo

As disciplinas cursadas pelo beneficiário ao longo do período de desenvolvimento deste relatório e suas respectivas notas finais, em parênteses, foram: Controle e Servomecanismo (8,8); Eletromagnetismo 2 (9,7); Engenharia de Dispositivos e Materiais Avançados (9,5); Engenharia Eletroquímica (7,6); Estado Sólido 1 (8,2); Sociologia Industrial e do Trabalho (7,8); Teoria da Informação: Clássica e Quântica (9,3); Teoria das Organizações (7,5).


\section*{Projeto inicial}
O projeto inicial até o momento da entrega do Relatório Parcial previu uma revisão de conceitos de Relatividade Especial, com análise dos princípios básicos da teoria e consequências, como as transformações de Lorentz e a dinâmica relativística, utilizando de referência \cite{Dray2012}.
Após isso, o estudo dos conceitos da Teoria da Relatividade Geral, começando por definições matemáticas básicas tais como o conceito de variedade e espaço-tempo curvo. Depois, a análise de campos tensoriais, derivadas e geodésicas no espaço curvo. Também propôs-se a definição dos símbolos de Christoffel, bem como os tensores de Riemann, de Ricci e de métrica. Em seguida, a introduzem-se conceitos físicos de modo a definir o tensor de energia-momento. Por fim, as equações de Einstein são apresentadas.  A referência principal utilizada foi \cite{Foster2006}, com as referencias complementares \cite{Hartle2005} e \cite{Schutz2009}. 

As referências sobre resultados de eletromagnetismo e mecânica clássica foram retiradas, respectivamente, de \cite{Griffiths2005} e \cite{nussenzveig2013curso}.
\\

Todas as figuras utilizadas neste relatório foram elaboradas pelo autor.