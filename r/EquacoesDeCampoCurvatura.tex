\chapter{Equações de campo e curvatura}\label{cap:EquacoesDeCampoCurvatura}
\section{Tensor de energia-momento e dinâmica de fluidos}\label{sec:EnergiaMomentoFluidos}
Tratamos, à princípio, de espaços planos, em referenciais inerciais, nesta seção. Utilizaremos 4-vetores e 3-vetores, simbolizando estes com o negrito. 

\[
\lambda^{\mu} \equiv\left(\lambda^{0}, \lambda^{1}, \lambda^{2}, \lambda^{3}\right) \equiv\left(\lambda^{0}, \boldsymbol{\lambda}\right)
\]

Considere uma partícula, com algumas quantidades relevantes:
\[\begin{array}{l}
{m \equiv \text{massa de repouso da partícula,}} \\
{t \equiv \text { tempo do referencial (tempo coordenado), }} \\ 
{\tau \equiv \text { tempo próprio da partícula, }} \\ 
{\gamma \equiv d t / d \tau=\left(1-v^{2} / c^{2}\right)^{-1 / 2}, \text { onde } v \text { é a velocidade da partícula, }} \\ 
{E \equiv \gamma m c^{2} \equiv \text { energia da partícula, }} \\ 
{u^{\mu} \equiv d x^{\mu} / d \tau \equiv \text { 4-velocidade da partícula }} \\ 
{v^{\mu} \equiv d x^{\mu} / d t \equiv u^{\mu} / \gamma \equiv \text { velocidade da partícula}} \\ 
{p^{\mu} \equiv m u^{\mu} \equiv 4 \text {-momento da partícula. }}
\end{array}\] 
Assim, na notação definida, $ v^\mu \equiv (c, \mathbf{v}) $, onde $\mathbf{v}$ é a 3-velocidade da partícula, de modo que o $v$ que aparece na fórmula de $\gamma$ é $|\mathbf{v}|$. 

Uma partícula estacionária situada em um ponto cujo vetor posição é $\mathbf{x}_0$ possui 4-velocidade
\[
u^\mu \equiv dx^\mu/d\tau = d(c\tau,\mathbf{x}_0)/d\tau = (c,\mathbf{0})
\]
e momento
\[
p^{\mu}=(c,\mathbf{0}).
\]
A zero-ésima componente de $p^\mu$ é, neste caso, a \textit{energia de repouso} a menos de um fator $c$. Para uma partícula em movimento, temos 
\begin{equation}\label{eq:QuadriMomentoParticula}
	p^\mu \equiv mu^\mu = \gamma m v^\mu = (\gamma mc, \gamma m\mathbf{v})=(E/c,\mathbf{p}).
\end{equation}

Vamos, agora, tratar de uma distribuição contínua de matéria. Por simplicidade, trabalharemos com um fluido perfeito, que é caracterizado por dois campos escalares, sua densidade $\rho$ e sua pressão $p$; e por um campo vetorial, sua 4-velocidade $u^{\mu}$. 

A fim de que $\rho$ seja um campo escalar, deve-se defini-lo de modo que seja a \textit{densidade própria}, isto é, a massa de repouso por unidade de volume de repouso. Aqui, no lugar do 4-momento da partícula $p^\mu\equiv mu^\mu$, temos a 4-densidade de momento $\rho u^\mu$.

Procura-se, então, um tensor de que, de alguma maneira, represente a energia do fluido e que, ao ser levado a um espaço-tempo curvo, possa agir como a origem do campo gravitacional. Na Relatividade Geral, perde-se a distinção de massa e energia, então toda forma de energia deveria produzir um campo gravitacional. Iremos simplesmente escrevê-lo e, então, interoretar seu significado físico. Ele é o chamado tensor de energia-momento e é definido como
\begin{equation}\label{eq:TensorEnergiaMomentoDefinicao}
	T^{\mu \nu} \equiv\left(\rho+p / c^{2}\right) u^{\mu} u^{\nu}-p \eta^{\mu \nu}.
\end{equation}
Pode-se notar que $T^{\mu \nu}$ é simétrico e depende dos campos vetoriais e escalares $\rho, p$ e $u^\mu$ que caracterizam o fluido. Além disso, temos que
\[
	T^{\mu \nu} u_{\nu}=c^{2}\left(\rho+p / c^{2}\right) u^{\mu}-p u^{\mu}=c^{2} \rho u^{\mu},	
\]
de modo que $T^{\mu \nu}{u_{\nu}}$ é, a menos de um fator $c^2$, a 4-densidade de momento do fluido. Finalmente, colocando sua divergência $T^{\mu\nu}_{,\mu}$ como zero nos dá duas equações importantes: a equação da continuidade e a equação de movimento. 

Fazendo a divergencia $T^{\mu\nu}_{,\mu}$ igual a zero nos diferenciá
\begin{equation}\label{eq:DivergenciaTensorEnergiaMomento}
	\left(\rho u^{\mu}\right)_{, \mu} u^{\nu}+\rho u^{\mu} u_{, \mu}^{\nu}+\left(p / c^{2}\right) u_{, \mu}^{\mu} u^{\nu}+\left(p / c^{2}\right) u^{\mu} u_{, \mu}^{\nu}+c^{-2} p_{, \mu} u^{\mu} u^{\nu}-p_{, \mu} \eta^{\mu \nu}=0.
\end{equation}
Notando que a 4-velocidade $u^\nu$ satisfaz $u^{\nu}u_\nu=c^2$, temos
\[
	u^\nu u_\nu = g_{\nu\sigma}u^\nu u^\sigma = c^2
\]
e, diferenciando {\color{ForestGreen} (notando que $g_{\nu\sigma;\mu}=0$)}, chegamos em
\[
	g_{\nu\sigma}\left(u^{\nu}_{;\mu}u^{\sigma}+u^{\nu}u^{\sigma}_{;\mu}\right)=0 
	\Rightarrow g_{\nu\sigma}u^{\nu}_{,\mu}u^{\sigma}+g_{\sigma\nu}u^{\sigma}_{,\mu}u^{\nu}= 2u_{,\mu}^{\nu} u_{\nu} = 0
\]
Assim, temos que $u_{,\mu}^{\nu} u_{\nu} = 0$. Contraindo, então, a equação \eqref{eq:DivergenciaTensorEnergiaMomento} com $u_\nu$ e dividindo por $c^2$ nos dá:
\begin{equation}\label{eq:EquacaoContinuidade4vetorial}
	\boxed{
		\left(\rho u^{\mu}\right)_{, \mu}+\left(p / c^{2}\right) u_{, \mu}^{\mu}=0.
	}
\end{equation}
Colocando esse resultado na equação \eqref{eq:DivergenciaTensorEnergiaMomento}, temos que ela simplifica para
\begin{equation}\label{eq:EquacaoMovimentoEuler4vetorial}
	\boxed{
		\left(\rho+p / c^{2}\right) u_{, \mu}^{\nu} u^{\mu}=\left(\eta^{\mu \nu}-c^{-2} u^{\mu} u^{\nu}\right) p_{, \mu}	.
	}
\end{equation}

Vamos mostrar, agora, que a equação \eqref{eq:EquacaoContinuidade4vetorial} e \eqref{eq:EquacaoMovimentoEuler4vetorial}, para fluidos em baixas velocidades e baixas pressões, reduzem para as equações clássicas de continuidade e de movimento de um fluido perfeito. Velocidades baixas implica que $\gamma=1$ e baixas pressões significa que $p/c^2 \ll \rho$. Assim, a equação  \eqref{eq:EquacaoContinuidade4vetorial}  reduz para
\[\left(\rho v^{\mu}\right)_{, \mu}=0,\]
o que dá
\[(\rho c)_{, 0}+\left(\rho v^{i}\right)_{, i}=0\]
e, em notação 3-vetorial isso é
\begin{equation}\label{eq:EquacaoContinuidadeClassica3vetorial}
	\boxed{
		\partial \rho / \partial t+\nabla \cdot(\rho \mathbf{v})=0,
	}
\end{equation}
que é a equação da continuidade clássica \textbf{(citar referência)}.

Já a equação \eqref{eq:EquacaoMovimentoEuler4vetorial}, nessa aproximação, reduz-se a 
\[
	\rho v_{, \mu}^{\nu} v^{\mu}=\left(\eta^{\mu \nu}-c^{-2} v^{\mu} v^{\nu}\right) p_{, \mu}
\]
e, para baixas velocidades, temos que
\[
\left[\eta^{\mu \nu}-c^{-2} v^{\mu} v^{\nu}\right] \approx
\left[\begin{array}{rrrr}{0} & {0} & {0} & {0} \\ {0} & {-1} & {0} & {0} \\ {0} & {0} & {-1} & {0} \\ {0} & {0} & {0} & {-1}\end{array}\right],
\]
de modo que as zero-ésimas componentes de ambos os lados da equação são zero. As outras componentes podem ser escritas como
\[
	\rho v^{i}, \mu^{\mu}= \rho\left[\partial v^{i} / \partial t+v_{, j}^{i} v^{j}\right]=-\delta^{j i} p_{, j} .\]
Em notação 3-vetorial, essa ultima igualdade é escrita como
\begin{equation}\label{eq:EquacaoMovimentoEulerClassica3vetorial}
	\boxed{
		\rho(\partial / \partial t+\mathbf{v} \cdot \nabla) \mathbf{v}=-\nabla p
	}
\end{equation}
que é a equação de movimento de Euler clássica para um fluido perfeito.

(pág. 100)




















\section{*Tensor de curvatura e tensores relacionados}\label{sec:TensorCurvatura}

Por questão de generalidade, esta seção tratará de variedades $N$-dimensionais, de modo que se utilizarão sufixos $a,b,\dots$, os quais vão de $1$ a $N$, em vez de $\mu,\nu,\dots$ que vão de $0$ a $3$. 

Vamos primeiramente estudar de perto uma propriedade da diferenciação covariante que difere da diferenciação parcial -- a ordem em que são feitas as operações importa, e mudar a ordem (em geral) muda o resultado.

A derivada covariante de um campo vetorial covariante $\lambda_a$ é
\[
\lambda_{a ; b} \equiv \partial_{b} \lambda_{a}-\Gamma_{a b}^{d} \lambda_{d},
\]
Uma segunda diferenciação nos dá 
\[
\begin{array}{c}{\lambda_{a ; b c}=\partial_{c}\left(\lambda_{a ; b}\right)-\Gamma_{a c}^{e} \lambda_{e ; b}-\Gamma_{b c}^{e} \lambda_{a ; e}} \\ {=\partial_{c} \partial_{b} \lambda_{a}-\left(\partial_{c} \Gamma_{a b}^{d}\right) \lambda_{d}-\Gamma_{a b}^{d} \partial_{c} \lambda_{d}-\Gamma_{a c}^{e}\left(\partial_{b} \lambda_{e}-\Gamma_{e b}^{d} \lambda_{d}\right)-\Gamma_{b c}^{e}\left(\partial_{e} \lambda_{a}-\Gamma_{a e}^{d} \lambda_{d}\right).}\end{array}
\]
Intercalando $b$ e $c$ e, então, subtraindo as expressões nos dá
\begin{equation}\label{eq:DiferencaDerivadaSegundaCovariante}
	\lambda_{a;bc}-\lambda_{a;cb}=R^d_{abc}\lambda_d,
\end{equation}
onde 
\begin{equation}\label{eq:DefinicaoTensorRiemann}
	\boxed{
		R_{a b c}^{d} \equiv \partial_{b} \Gamma_{a c}^{d}-\partial_{c} \Gamma_{a b}^{d}+\Gamma_{a c}^{e} \Gamma_{e b}^{d}-\Gamma_{a b}^{e} \Gamma_{e c}^{d}.
	}
\end{equation}

Como o lado esquerdo da equação \eqref{eq:DiferencaDerivadaSegundaCovariante} é um tensor para vetores $\lambda_a$ arbitrários, a contração de $R^d_{abc}$ com $\lambda_d$ também é um tensor e, como $R^d_{abc}$ não depende de $\lambda_a$, o Teorema do Quociente leva-nos a concluir que $R^d_{abc}$ é um tensor do tipo $(1,3)$. Denomina-se tensor de curvatura (ou tensor de Riemann).

Então, a condição necessária e suficiente para que a ordem em que a diferenciação covariante de um tensor $(0,1)$ qualquer pode ser trocada é que $R^a_{bcd}=0$.

{\color{red} Fazer exercício 3.2.1 e afirmar que é válido para qualquer campo}

No espaço-tempo plano da Relatividade Especial, sabemos que existem referenciais em que $g_{\mu\nu}=\eta_{\mu\nu}$ e que, nesses sistemas coordenados, $\Gamma^\mu_{\nu\sigma}=0$ e, então, o tensor de curvatura é identicamente nulo.
 %mas isso nem sempre acontece blabla
 {\color{red} Problema 3.1 como exemplo}

 Podemos, agora, formalizar a definição de {\color{ForestGreen}planicidade}. Uma variedade (ou uma região de uma variedade) é \textit{plana} se, em todos os seus pontos, $R^a_{bcd}=0$; de outro modo, ela é \textit{curvada}. 

 A princípio, o tensor $R^a_{bcd}$ possui $N^4$ componentes. No entanto, ele possui muitas simetrias e suas componentes satisfazem uma identidade importante, de modo que esse número abaixa para $N^2(N^2-1)/12$ componentes independentes. 
 
 A partir da definição \eqref{eq:DefinicaoTensorRiemann}, temos que
\begin{align*}
	&R_{b c d}^{a}+R_{c d b}^{a}+R_{d b c}^{a}= 
	\partial_{c} \Gamma_{bd}^{a}-\partial_{d} \Gamma_{bc}^{a}+\Gamma_{bd}^{e}
	\Gamma_{ce}^{a}-\Gamma_{bc}^{e} \Gamma_{de}^{a} + \\
	+&\partial_{d} \Gamma_{bc}^{a}-\partial_{b} \Gamma_{cd}^{a}+\Gamma_{ab}^{e} \Gamma_{de}^{a}-\Gamma_{cd}^{e} \Gamma_{be}^{a} +
	\partial_{b} \Gamma_{cd}^{a}-\partial_{c} \Gamma_{bd}^{a}+\Gamma_{cd}^{e} \Gamma_{be}^{a}-\Gamma_{bd}^{e} \Gamma_{ce}^{a}=0,
\end{align*}
ou, resumindo, temos a relação:
 \begin{equation}\label{eq:IdentidadeCiclicaTensorDeCurvatura}
	 R^a_{bcd}+R^a_{cdb}+R^a_{dbc}=0.
 \end{equation}
 Chama-se \textit{identidade cíclica}. As simetrias desse tensor são mais facilmente expressas em termos do tensor associado do tipo $(0,4)$
\[
	R_{abcd} \equiv g_{ae}R^e_{bcd}.
\]
{\color{red}
A partir das equações 2.33 2.35 e 2.36 chegamos em
}

\begin{equation}
	R_{a b c d} \equiv \tfrac{1}{2}\left(\partial_{d} \partial_{a} g_{b c}-\partial_{d} \partial_{b} g_{a c}+\partial_{c} \partial_{b} g_{a d}-\partial_{c} \partial_{a} g_{b d}\right)-g^{e f}\left(\Gamma_{e a c} \Gamma_{f b d}-\Gamma_{e a d} \Gamma_{f b c}\right)
\end{equation}
A partir disso, apenas é necessário checarem-se as seguintes propriedades:

\begin{equation}\label{eq:RSimetriaA}
	R_{abcd}=-R_{bacd}
\end{equation}
\begin{equation}\label{eq:RSimetriaB}
	R_{abcd}=-R_{abdc}
\end{equation}
\begin{equation}\label{eq:RSimetriaC}
	R_{abcd}=R_{cdab}
\end{equation}
Segue de \eqref{eq:RSimetriaA} que
\[
R^a_{acd} = 0.
\]


\begin{proof}
	Provar relações 3.16-3.17
\end{proof}

Considere, agora, um ponto $P$ qualquer e um sistema de coordenadas tal que $(T^{a}_{bc})_P=0$, {\color{ForestGreen} como explicado na seção \ref{sec:CoordenadasGeodesicas}.} Ao diferenciar a equação \eqref{eq:DefinicaoTensorRiemann} e aplicá-la ao ponto $P$, nesse sistema de coordenadas, temos
\[
	\left(R_{b c d ; e}^{a}\right)_{\mathrm{P}}=\left(\partial_{e} \partial_{c} \Gamma_{b d}^{a}-\partial_{e} \partial_{d} \Gamma_{b c}^{a}\right)_{\mathrm{P}} .
\]
Ao permutarmos $c,d,e$ e somando o resultado, chegamos na \textit{identidade de Bianchi}:
\begin{equation}\label{eq:IdentidadeBianchi}
	R_{b c d ; e}^{a}+R_{b d e ; c}^{a}+R_{b e c ; d}^{a}=0,
\end{equation}
em $P$, mas como este é um ponto genérico, o resultado vale sempre.

Um outro tensor importante é o \textit{tensor de Ricci}, que também utiliza a mesma letra de base do tensor de curvatura. Ele é dado pela contração 
\begin{equation}\label{eq:TensorRicciDefinicao}
	\boxed{
	R_{ab} \equiv R^c_{abc} .
	}
\end{equation}

Contraindo $a$ na identidade cíclica \eqref{eq:IdentidadeCiclicaTensorDeCurvatura}, temos
\[
	0=R_{b c a}^{a}+R_{c a b}^{a}+R_{a b c}^{a}=R_{b c}-R_{c b a}^{a}+0=R_{b c}-R_{c b},
\]
o que nos diz que o tensor de Ricci também é simétrico. Como $R_{ab}$ é simétrico, $\tensor{R}{^a _b}=\tensor{R}{_b ^a}$ e podemos denotar ambos por $R^a_b$. Mais uma contração nos dá o \textit{escalar de curvatura}
\begin{equation}\label{eq:EscalarCurvaturaDefinicao}
	\boxed{
		R \equiv g^{ab}R_{ab}=R^a_a.
	}
\end{equation}

Um último tensor importante é o \textit{tensor de Einstein}, definido por
\begin{equation}
	\boxed{
		G_{ab} \equiv R_{ab} - \tfrac{1}{2}Rg_{ab} .
	}
\end{equation}

{\color{red} falta mostrar que o tensor de Einstein tem divergencia nula}






\section{*Curvatura e transporte paralelo}\label{sec:CurvaturaTransporteParalelo}

Discutiu-se na Seção \ref{sec:TransporteParalelo} que o transporte paralelo em uma variedade curvada dependia do caminho, mostrando especificamente que esse era o caso para uma esfera. No entanto, nós agora formalizamos o conceito de curvatura em termos do tensor de Riemann. Nesta seção, o intuito é deixar clara a conexão entre esse o transporte paralelo e o tensor de curvatura. Mais especificamente, mostraremos como a mudança $\Delta\lambda^a$ que resulta do transporte paralelo de um vetor $\lambda^a$ ao longo de uma pequena curva fechada ao retor de um ponto $P$ se relaciona com o tensor de curvatora em $P$.

Suponha, por exemplo, que $\lambda^a$ é transportado paralelamente ao longo de uma curva $\gamma$ de um ponto inicial $O$ em que suas componentes são $\lambda_0^a$. Se $\gamma$ é parametrizado por $t$, então $\lambda^a$ satisfaz a equação diferencial
\begin{equation}\label{eq:TransporteParaleloDiferencial}
	\frac{d\lambda^a}{dt} = -\Gamma^a_{bc}\lambda^b\frac{dx^c}{dt},
\end{equation}
de onde podemos chegar que $\lambda$ satisfaz a equação integral
\begin{equation}\label{eq:TransporteParaleloIntegral}
	\lambda^a = \lambda^a_0 - \int\Gamma^a_{bc}\lambda^b dx^c,
\end{equation}
onde a integral é tomada ao longo de $\gamma$ do ponto inicial $O$. Assim, podemos calcular a mudança $\Delta\lambda^a$ à medida que $\lambda$ é transportado ao longo de uma curva pequena próxima a $P$. Se $P$ possui coordenadas $x^a_P$, então os pontos na curva possuirão coordenadas $x^a$ dadas por
\[
	x^a=x^a_P+\xi^a
\]
onde os $\xi^a$ são pequenos. As diferenças entre coordenadas $\xi^a$ podem ser pensadas como um vetor partindo de $P$ para um ponto qualquer de $\gamma$. (veja a Figura \ref{fig:SmallLoop}). Como os $x^a_P$ são constantes, a equação \eqref{eq:TransporteParaleloIntegral} pode ser escrita como
\begin{equation}\label{eq:TransporteParaleloIntegralXi}
	\lambda^a=\lambda^a_0 - \int\Gamma^a_{bc}\lambda^b d\xi^c .
\end{equation}

\begin{figure}[t]
	\centering
	\plot{0.4\linewidth}{figuras/SmallLoop}
	\caption{Uma curva pequena ao redor de $P$.}
	\label{fig:SmallLoop}
\end{figure}

Apesar de termos chegado em uma relação entre $\lambda^a$ e $\lambda^a_0$, nós não podemos calculá-la de uma maneira direita, uma vez que o vetor transportado também aparece na integral à direita. No entanto, podemos utilizar essa relação para conseguir aproximações cada vez melhores que são válidas quando $\lambda^a$ não muda muito de seu valor inicial $\lambda^a_0$, que será o caso para nossa curva pequena próxima de $P$.

Como uma primeira aproximação, podemos utilizar $\lambda^b = \lambda^b_0$ na integral ao lado direito e, então, podemos utilizar o resultado como uma aproximação melhor.
\begin{align*}
	\lambda^a &= \lambda^a_0 - \int\Gamma^a_{bc}\lambda^a_0 d\xi^c \\
	&= \lambda^a_0 - \lambda^b_0\int\Gamma^a_{bc}d\xi^c .
\end{align*}
Utilizando esse resultado ao lado direito, podemos chegar em uma aproximação ainda melhor:
\begin{equation}\label{eq:AproximacaoSegundaOrdemLambda}
\begin{aligned} \lambda^{a} &=\lambda_{0}^{a}-\int \Gamma_{b c}^{a}\left(\lambda_{0}^{b}-\lambda_{0}^{d} \int \Gamma_{d e}^{b} d \xi^{e}\right) d \xi^{c} \\ &=\lambda_{0}^{a}-\lambda_{0}^{b} \int \Gamma_{b c}^{a} d \xi^{c}+\lambda_{0}^{d} \int \Gamma_{b c}^{a}\left(\int \Gamma_{d e}^{b} d \xi^{e}\right) d \xi^{c}. \end{aligned}
\end{equation}
Esse processo pode ser repetido indefinidamente, mas a aproximação dada por \eqref{eq:AproximacaoSegundaOrdemLambda} é o suficiente para o nosso propósito, o qual envolve utilizar aproximações de segunda ordem em $\xi^a.$.







% EXEMPLO DE LOOP NA ESFERA
% \begin{figure}[t]
% 	\centering
% 	\plot{0.4\linewidth}{figuras/EsferaLoop}
% 	\caption{A small loop on a shpere.}
% 	\label{fig:EsferaLoop}
% \end{figure}


\section{*Desvio geodésico}\label{sec:DesvioGeodesico}


\begin{figure}[t]
	\centering
	\plot{0.4\linewidth}{figuras/DesvioGeodesico}
	\caption{Geodesic deviation.}
	\label{fig:DesvioGeodesico}
\end{figure}

\section{*Equações de campo de Einstein}\label{sec:EquacoesCampoEinstein}

\section{*Equações de Einstein e de Poisson}\label{sec:EinsteinPoisson}














\section{*A solução de Schwarzschild}\label{sec:Schwarzschild}

As equações de campo discutidas aqui são extremamentes difíceis de serem resolvidas, visto que possuem um alto grau de não-linearidade. Assim, para encontrar soluções, é preciso simplificar o problema, isto é, buscar situações de alta simetria.

A primeira solução das equações de campo foi uma solução especial e foi obtida por K. Schwarzschild em 1916. O que foi procurado foi o campo tensorial de métrica representando um campo gravitacional estático e de simetria esférica situado no espaço-tempo vazio em torno de um objeto esférico massivo, tal como uma estrela. Suas suposições para o problema foram:
%  {\color{red}(comentar sobre a condição (a) ser redundante?)}
\begin{enumerate}[label=(\alph*)]
	\item o campo é estático,\label{sup:estatico}
	\item o campo possui simetria esférica,\label{sup:esferico}
	\item o espaço-tempo está vazio, \label{sup:vazio}
	\item o espaço-tempo é assintoticamente plano.\label{sup:plano}
\end{enumerate}

Também assumiu-se que o espaço-tempo poderia ser descrito por um sistema de coordenadas $(t,r,\theta,\phi)$, onde $t$ é uma {\color{red} coordenada do tipo tempo}, % $\theta$ e $\phi$ são os ângulos polares blabalbla??

Então, procurou-se um elemento de linha da forma
\begin{equation}\label{eq:ElementoLinhaSchwarzschildPostulado}
	c^2d\tau^2=A(r)dt^2-B(r)dr^2-r^2d\theta^2-r^2\sin^2\theta d\phi^2,
\end{equation}
onde $A(r)$ e $B(r)$ são funções deseconhecidas de $r$ a serem obtidas ao resolver as equações.

O fato que nenhum dos termos depende de $t$ reflete a suposição \ref{sup:estatico}, e o fato que as superfícies dadas por $r,t$ constantes têm elementos de linha dados por
\begin{equation}\label{eq:RTConstantes}
	d s^{2}=r^{2}\left(d \theta^{2}+\sin ^{2} \theta d \phi^{2}\right)
\end{equation}
mostra que possuem a geometria de esferas {\color{red}(Exercício 1.6.2)} e isso indica sua suposição \ref{sup:esferico}. A suposição \ref{sup:vazio} significa que $A(r), B(r)$ devem ser encontrados utilizando as equações de campo do espaço-tempo vazio $R_{\mu\nu}=0$, ao passo que a suposição \ref{sup:plano} nos dá as condições de contorno em $A,B$:

\begin{equation}\label{eq:CondicoesContornoSchwarzschild}
	A(r) \rightarrow c^{2} \qq{ e } B(r) \rightarrow 1 \text { quando } r \rightarrow \infty
\end{equation}

{\color{red} $r$ não é a distância radial???}

Vamos agora extrair a solução de Schwarzschild das equações de campo. O intuito é utilizar o tensor de métrica $g_{\mu\nu}$ obtido pelo elemento de linha \refeq{eq:ElementoLinhaSchwarzschildPostulado} como uma tentativa de solução da equação de campo para o espaço-tempo vazio. Da equação \textbf{eq:DefinicaoTensorRicci}, temos 
\[
R_{\mu \nu} \equiv \partial_{\nu} \Gamma_{\mu \sigma}^{\sigma}-\partial_{\sigma} \Gamma_{\mu \nu}^{\sigma}+\Gamma_{\mu \sigma}^{\rho} \Gamma_{\rho \nu}^{\sigma}-\Gamma_{\mu \nu}^{\rho} \Gamma_{\rho \sigma}^{\sigma}
\]
e do exemplo {\color{red} FAZER PROBLEMA 2.7}, temos
\[
\begin{array}{lll}
{\Gamma_{01}^{0}=A^{\prime} / 2 A,} & {\Gamma_{00}^{1}=A^{\prime} / 2 B,} & {\Gamma_{11}^{1}=B^{\prime} / 2 B,} \\ 
{\Gamma_{22}^{1}=-r / B,} & {\Gamma_{33}^{1}=-\left(r \sin ^{2} \theta\right) / B,} &{ \Gamma_{12}^{2}=1 / r,} \\ 
{\Gamma_{33}^{2}=-\sin \theta \cos \theta,} & {\Gamma_{13}^{3}=1 / r,} & {\Gamma_{23}^{3}=\cot \theta,}
\end{array}	
\]
com todos os outros coeficientes de conexão nulos. Aqui fizemos a identificação das coordenadas $x^0\equiv t, x^1\equiv r, x^2\equiv\theta,x^3\equiv\phi$ e a linha significa uma diferenciação com respeito a $r$. 
{\color{red} fazer as contas, a partir de $R_{\mu\nu}=0$}, temos

\begin{equation}\label{eq:SolucaoR00}
	R_{00}=-\frac{A^{\prime \prime}}{2 B}+\frac{A^{\prime}}{4 B}\left(\frac{A^{\prime}}{A}+\frac{B^{\prime}}{B}\right)-\frac{A^{\prime}}{r B}=0,
\end{equation}
\begin{equation}\label{eq:SolucaoR11}
	R_{11}=\frac{A^{\prime \prime}}{2 A}-\frac{A^{\prime}}{4 A}\left(\frac{A^{\prime}}{A}+\frac{B^{\prime}}{B}\right)-\frac{B^{\prime}}{r B}=0
\end{equation}
\begin{equation}\label{eq:SolucaoR22}
	R_{22}=\frac{1}{B}-1+\frac{r}{2 B}\left(\frac{A^{\prime}}{A}-\frac{B^{\prime}}{B}\right)=0
\end{equation}
\begin{equation}\label{eq:SolucaoR33}
	R_{33}=R_{22} \sin ^{2} \theta=0
\end{equation}
e $R_{\mu\nu}=0$ sem nenhuma condição a mais para $\mu\neq\nu$