\chapter{Equações de campo e curvatura}\label{cap:EquacoesDeCampoCurvatura}
\section{Tensor de energia-momento e dinâmica de fluidos}\label{sec:EnergiaMomentoFluidos}
Trataremos, à princípio, de espaços planos, em referenciais inerciais. Utilizaremos 4-vetores e 3-vetores, simbolizando estes com o negrito. 

\[
\lambda^{\mu} \equiv\left(\lambda^{0}, \lambda^{1}, \lambda^{2}, \lambda^{3}\right) \equiv\left(\lambda^{0}, \boldsymbol{\lambda}\right)
\]

Considere uma partícula, com algumas quantidades relevantes:
\begin{align*}
	m 		&\equiv \text{massa de repouso da partícula,} \\
	t 		&\equiv \text { tempo do referencial (tempo coordenado),} \\ 
	\tau 	&\equiv \text { tempo próprio da partícula, } \\ 
	\gamma 	&\equiv d t / d \tau=\left(1-v^{2} / c^{2}\right)^{-1 / 2}, \text { onde } v \text { é a velocidade da partícula, } \\ 
	E 		&\equiv \gamma m c^{2} \equiv \text { energia da partícula, } \\ 
	u^{\mu} &\equiv d x^{\mu} / d \tau \equiv \text { 4-velocidade da partícula } \\ 
	v^{\mu} &\equiv d x^{\mu} / d t \equiv u^{\mu} / \gamma \equiv \text { velocidade da partícula} \\ 
	p^{\mu} &\equiv m u^{\mu} \equiv 4 \text {-momento da partícula.}
\end{align*}
Assim, na notação definida, $ v^\mu \equiv (c, \mathbf{v}) $, onde $\mathbf{v}$ é a 3-velocidade da partícula, de modo que o $v$ que aparece na fórmula de $\gamma$ é $|\mathbf{v}|$. 

Uma partícula estacionária situada em um ponto cujo vetor posição é $\mathbf{x}_0$ possui 4-velocidade
\[
u^\mu \equiv dx^\mu/d\tau = d(c\tau,\mathbf{x}_0)/d\tau = (c,\mathbf{0})
\]
e 4-momento
\[
p^{\mu}=(mc,\mathbf{0}).
\]
A zero-ésima componente de $p^\mu$ é, neste caso, a \textit{energia de repouso} a menos de um fator $c$. Para uma partícula em movimento, temos 
\begin{equation}\label{eq:QuadriMomentoParticula}
	p^\mu \equiv mu^\mu = \gamma m v^\mu = (\gamma mc, \gamma m\mathbf{v})=(E/c,\mathbf{p}).
\end{equation}

Vamos, agora, tratar de uma distribuição contínua de matéria. Por simplicidade, trabalharemos com um fluido perfeito, que é caracterizado por dois campos escalares, sua densidade $\rho$ e sua pressão $p$; e por um campo vetorial, sua 4-velocidade $u^{\mu}$. 

A fim de que $\rho$ seja um campo escalar, deve-se defini-lo de modo que seja a \textit{densidade própria}, isto é, a massa de repouso por unidade de volume de repouso. Aqui, no lugar do 4-momento da partícula $p^\mu\equiv mu^\mu$, temos a 4-densidade de momento $\rho u^\mu$.

Procura-se, então, um tensor de que, de alguma maneira, represente a energia do fluido e que, ao ser levado a um espaço-tempo curvo, possa agir como a origem do campo gravitacional. Na Relatividade Geral, perde-se a distinção de massa e energia, de modo que toda forma de energia deveria produzir um campo gravitacional. Iremos simplesmente escrevê-lo e interpretar seu significado físico. Ele é o chamado tensor de energia-momento e é definido como
\begin{equation}\label{eq:TensorEnergiaMomentoDefinicao}
	T^{\mu \nu} \equiv\left(\rho+p / c^{2}\right) u^{\mu} u^{\nu}-p \eta^{\mu \nu}.
\end{equation}
Pode-se notar que $T^{\mu \nu}$ é simétrico e depende dos campos vetoriais e escalares $\rho, p$ e $u^\mu$ que caracterizam o fluido. Além disso, temos que
\[
	T^{\mu \nu} u_{\nu}=c^{2}\left(\rho+p / c^{2}\right) u^{\mu}-p u^{\mu}=c^{2} \rho u^{\mu},	
\]
de modo que $T^{\mu \nu}{u_{\nu}}$ é, a menos de um fator $c^2$, a 4-densidade de momento do fluido. Finalmente, colocando sua divergência $\tensor{T}{^{\mu\nu}_{,\mu}}$ como zero nos dá duas equações importantes: a equação da continuidade e a equação de movimento. 

Fazendo a divergencia $\tensor{T}{^{\mu\nu}_{,\mu}}$ igual a zero nos diferenciá
\begin{equation}\label{eq:DivergenciaTensorEnergiaMomento}
	\left(\rho u^{\mu}\right)_{, \mu} u^{\nu}+\rho u^{\mu} u_{, \mu}^{\nu}+\left(p / c^{2}\right) u_{, \mu}^{\mu} u^{\nu}+\left(p / c^{2}\right) u^{\mu} u_{, \mu}^{\nu}+c^{-2} p_{, \mu} u^{\mu} u^{\nu}-p_{, \mu} \eta^{\mu \nu}=0.
\end{equation}
Notando que a 4-velocidade $u^\nu$ satisfaz $u^{\nu}u_\nu=c^2$, temos
\[
	u^\nu u_\nu = g_{\nu\sigma}u^\nu u^\sigma = c^2
\]
se, diferenciando (notando que $g_{\nu\sigma;\mu}=0$), chegamos em
\[
	g_{\nu\sigma}\left(u^{\nu}_{;\mu}u^{\sigma}+u^{\nu}u^{\sigma}_{;\mu}\right)=0 
	\Rightarrow g_{\nu\sigma}u^{\nu}_{,\mu}u^{\sigma}+g_{\sigma\nu}u^{\sigma}_{,\mu}u^{\nu}= 2u_{,\mu}^{\nu} u_{\nu} = 0
\]
Assim, temos que $u_{,\mu}^{\nu} u_{\nu} = 0$. Contraindo, então, a equação \eqref{eq:DivergenciaTensorEnergiaMomento} com $u_\nu$ e dividindo por $c^2$ nos dá:
\begin{equation}\label{eq:EquacaoContinuidade4vetorial}
	\boxed{
		\left(\rho u^{\mu}\right)_{, \mu}+\left(p / c^{2}\right) u_{, \mu}^{\mu}=0.
	}
\end{equation}
Colocando esse resultado na equação \eqref{eq:DivergenciaTensorEnergiaMomento}, temos que ela simplifica para
\begin{equation}\label{eq:EquacaoMovimentoEuler4vetorial}
	\boxed{
		\left(\rho+p / c^{2}\right) u_{, \mu}^{\nu} u^{\mu}=\left(\eta^{\mu \nu}-c^{-2} u^{\mu} u^{\nu}\right) p_{, \mu}	.
	}
\end{equation}

Vamos mostrar, agora, que as equações \eqref{eq:EquacaoContinuidade4vetorial} e \eqref{eq:EquacaoMovimentoEuler4vetorial}, para fluidos em baixas velocidades e baixas pressões, se reduzem ns equações clássicas de continuidade e de movimento de um fluido perfeito. Velocidades baixas implica que $\gamma\approx1$ e baixas pressões significa que $p/c^2 \ll \rho$. Assim, a equação  \eqref{eq:EquacaoContinuidade4vetorial} se reduz em
\[\left(\rho v^{\mu}\right)_{, \mu}=0,\]
o que dá
\[(\rho c)_{, 0}+\left(\rho v^{i}\right)_{, i}=0\]
e, em notação 3-vetorial isso é
\begin{equation}\label{eq:EquacaoContinuidadeClassica3vetorial}
	\boxed{
		\partial \rho / \partial t+\nabla \cdot(\rho \mathbf{v})=0,
	}
\end{equation}
que é a equação da continuidade clássica.

Já a equação \eqref{eq:EquacaoMovimentoEuler4vetorial}, nessa aproximação, reduz-se a 
\[
	\rho v_{, \mu}^{\nu} v^{\mu}=\left(\eta^{\mu \nu}-c^{-2} v^{\mu} v^{\nu}\right) p_{, \mu}
\]
e, para baixas velocidades, temos que
{\small
\[
\left[\eta^{\mu \nu}-c^{-2} v^{\mu} v^{\nu}\right] \approx
\left[\begin{array}{rrrr}{0} & {0} & {0} & {0} \\ {0} & {-1} & {0} & {0} \\ {0} & {0} & {-1} & {0} \\ {0} & {0} & {0} & {-1}\end{array}\right],
\]
}
de modo que as zero-ésimas componentes de ambos os lados da equação são zero. As outras componentes podem ser escritas como
\[
	\rho v^{i}_{, \mu}v^{\mu}= \rho\left[\partial v^{i} / \partial t+v_{, j}^{i} v^{j}\right]=-\delta^{j i} p_{, j} .\]
Em notação 3-vetorial, essa ultima igualdade é escrita como
\begin{equation}\label{eq:EquacaoMovimentoEulerClassica3vetorial}
	\boxed{
		\rho(\partial / \partial t+\mathbf{v} \cdot \nabla) \mathbf{v}=-\nabla p
	}
\end{equation}
que é a equação de movimento de Euler clássica para um fluido perfeito.

Se nós aceitássemos a equação \eqref{eq:EquacaoContinuidade4vetorial} como a equação relativística de continuidade e \eqref{eq:EquacaoMovimentoEuler4vetorial} como a equação de movimento de um fluido perfeito, então poderiámos inverter nosso argumento, e afirmar que $\tensor{T}{^{\mu\nu}_{,\mu}}=0$ \textit{por causa} das equações de continuidade e de movimento.

Vamos, agora, começar a trabalhar com o espaço-tempo curvo da Relatividade Geral. A seção \ref{sec:EspacotempoRelatividadeGeral} nos dá um modo de expandir definições e equações tensoriais do espaço-tempo plano. Em particular, trocando $\eta_{\mu\nu}$ por $g_{\mu\nu}$, e derivadas parciais por covariantes, nossa equação de definição do tensor de energia-momento de um fluido perfeito se torna
\begin{equation}\label{eq:TensorEnergiaMomentoCurvo}
	\boxed{
	T^{\mu\nu} \equiv (\rho+p/c^2)u^\mu u^\nu - pg^{\mu\nu} ,
	}
\end{equation}
e 
\begin{equation}\label{eq:DivergenciaTensorEnergiaMomentoCurvo}
	\boxed{ 
	\tensor{T}{^\mu^\nu_{;\mu}}=0 .
	} 
\end{equation}
Com definições adequadas do tensor de energia-momento, a equação \eqref{eq:DivergenciaTensorEnergiaMomentoCurvo} é válida para todos os fluidos e campos, não só para fluidos perfeitos. 

\section{Tensor de curvatura e tensores relacionados}\label{sec:TensorCurvatura}

Por questão de generalidade, esta seção tratará de variedades $N$-dimensionais, de modo que se utilizarão sufixos $a,b,\dots$, os quais vão de $1$ a $N$, em vez de $\mu,\nu,\dots$ que vão de $0$ a $3$. 

Vamos primeiramente estudar de perto uma propriedade da diferenciação covariante que difere da diferenciação parcial -- a ordem em que são feitas as operações importa, e mudar a ordem (em geral) muda o resultado.

A derivada covariante de um campo vetorial covariante $\lambda_a$ é
\[
\lambda_{a ; b} \equiv \partial_{b} \lambda_{a}-\Gamma_{a b}^{d} \lambda_{d},
\]
Uma segunda diferenciação nos dá 
\[
\begin{array}{c}{\lambda_{a ; b c}=\partial_{c}\left(\lambda_{a ; b}\right)-\Gamma_{a c}^{e} \lambda_{e ; b}-\Gamma_{b c}^{e} \lambda_{a ; e}} \\ {=\partial_{c} \partial_{b} \lambda_{a}-\left(\partial_{c} \Gamma_{a b}^{d}\right) \lambda_{d}-\Gamma_{a b}^{d} \partial_{c} \lambda_{d}-\Gamma_{a c}^{e}\left(\partial_{b} \lambda_{e}-\Gamma_{e b}^{d} \lambda_{d}\right)-\Gamma_{b c}^{e}\left(\partial_{e} \lambda_{a}-\Gamma_{a e}^{d} \lambda_{d}\right).}\end{array}
\]
Intercalando $b$ e $c$ e, então, subtraindo as expressões nos dá
\begin{equation}\label{eq:DiferencaDerivadaSegundaCovariante}
	\lambda_{a;bc}-\lambda_{a;cb}=R^d_{abc}\lambda_d,
\end{equation}
onde 
\begin{equation}\label{eq:DefinicaoTensorRiemann}
	\boxed{
		R_{a b c}^{d} \equiv \partial_{b} \Gamma_{a c}^{d}-\partial_{c} \Gamma_{a b}^{d}+\Gamma_{a c}^{e} \Gamma_{e b}^{d}-\Gamma_{a b}^{e} \Gamma_{e c}^{d}.
	}
\end{equation}

Como o lado esquerdo da equação \eqref{eq:DiferencaDerivadaSegundaCovariante} é um tensor para vetores $\lambda_a$ arbitrários, a contração de $R^d_{abc}$ com $\lambda_d$ também é um tensor e, como $R^d_{abc}$ não depende de $\lambda_a$, o Teorema do Quociente leva-nos a concluir que $R^d_{abc}$ é um tensor do tipo $(1,3)$. Denomina-se tensor de curvatura (ou tensor de Riemann).

Então, a condição necessária e suficiente para que a ordem em que a diferenciação covariante de um tensor $(0,1)$ qualquer pode ser trocada é que $R^a_{bcd}=0$.


No espaço-tempo plano da Relatividade Especial, sabemos que existem referenciais em que $g_{\mu\nu}=\eta_{\mu\nu}$ e que, nesses sistemas coordenados, $\Gamma^\mu_{\nu\sigma}=0$ e, então, o tensor de curvatura é identicamente nulo.

 Podemos, agora, formalizar a definição de planicidade. Uma variedade (ou uma região de uma variedade) é \textit{plana} se, em todos os seus pontos, $R^a_{bcd}=0$; de outro modo, ela é \textit{curva}. 

 A princípio, o tensor $R^a_{bcd}$ possui $N^4$ componentes. No entanto, ele possui muitas simetrias e suas componentes satisfazem uma identidade importante, de modo que esse número abaixa para $N^2(N^2-1)/12$ componentes independentes. 
 
 A partir da definição \eqref{eq:DefinicaoTensorRiemann}, temos que
\begin{align*}
	&R_{b c d}^{a}+R_{c d b}^{a}+R_{d b c}^{a}= 
	\partial_{c} \Gamma_{bd}^{a}-\partial_{d} \Gamma_{bc}^{a}+\Gamma_{bd}^{e}
	\Gamma_{ce}^{a}-\Gamma_{bc}^{e} \Gamma_{de}^{a} + \\
	+&\partial_{d} \Gamma_{bc}^{a}-\partial_{b} \Gamma_{cd}^{a}+\Gamma_{ab}^{e} \Gamma_{de}^{a}-\Gamma_{cd}^{e} \Gamma_{be}^{a} +
	\partial_{b} \Gamma_{cd}^{a}-\partial_{c} \Gamma_{bd}^{a}+\Gamma_{cd}^{e} \Gamma_{be}^{a}-\Gamma_{bd}^{e} \Gamma_{ce}^{a}=0,
\end{align*}
ou, resumindo, temos a relação:
 \begin{equation}\label{eq:IdentidadeCiclicaTensorDeCurvatura}
	 R^a_{bcd}+R^a_{cdb}+R^a_{dbc}=0.
 \end{equation}
 Chama-se \textit{identidade cíclica}. As simetrias desse tensor são mais facilmente expressas em termos do tensor associado do tipo $(0,4)$
\[
	R_{abcd} \equiv g_{ae}R^e_{bcd}.
\]

A partir das equações \eqref{eq:ChristoffelPrimeiroTipo}, \eqref{eq:LevantarIndiceGamma} e \eqref{eq:AbaixarIndiceGamma} chegamos em
\begin{equation}
	R_{a b c d} \equiv \tfrac{1}{2}\left(\partial_{d} \partial_{a} g_{b c}-\partial_{d} \partial_{b} g_{a c}+\partial_{c} \partial_{b} g_{a d}-\partial_{c} \partial_{a} g_{b d}\right)-g^{e f}\left(\Gamma_{e a c} \Gamma_{f b d}-\Gamma_{e a d} \Gamma_{f b c}\right)
\end{equation}
A partir dessa relação, obtemos as seguintes propriedades:
\begin{equation}\label{eq:RSimetriaA}
	R_{abcd}=-R_{bacd}
\end{equation}
\begin{equation}\label{eq:RSimetriaB}
	R_{abcd}=-R_{abdc}
\end{equation}
\begin{equation}\label{eq:RSimetriaC}
	R_{abcd}=R_{cdab}
\end{equation}
Segue de \eqref{eq:RSimetriaA} (e da simetria do tensor de métrica) que
\[
R^a_{acd} = 0.
\]


Considere, agora, um ponto $P$ qualquer e um sistema de coordenadas tal que $(\Gamma^{a}_{bc})_P=0$, como explicado na seção \ref{sec:CoordenadasGeodesicas}. Ao diferenciar a equação \eqref{eq:DefinicaoTensorRiemann} e aplicá-la ao ponto $P$, nesse sistema de coordenadas, temos
\[
	\left(R_{b c d ; e}^{a}\right)_{\mathrm{P}}=\left(\partial_{e} \partial_{c} \Gamma_{b d}^{a}-\partial_{e} \partial_{d} \Gamma_{b c}^{a}\right)_{\mathrm{P}} .
\]
Ao permutarmos $c,d,e$ e somando o resultado, chegamos na \textit{identidade de Bianchi}:
\begin{equation}\label{eq:IdentidadeBianchi}
	R_{b c d ; e}^{a}+R_{b d e ; c}^{a}+R_{b e c ; d}^{a}=0,
\end{equation}
em $P$, mas como este é um ponto genérico, o resultado vale sempre.

Um outro tensor importante é o \textit{tensor de Ricci}, que também utiliza a mesma letra de base do tensor de curvatura. Ele é dado pela contração 
\begin{equation}\label{eq:TensorRicciDefinicao}
	\boxed{
	R_{ab} \equiv R^c_{abc} .
	}
\end{equation}

Contraindo $a$ na identidade cíclica \eqref{eq:IdentidadeCiclicaTensorDeCurvatura}, temos
\[
	0=R_{b c a}^{a}+R_{c a b}^{a}+R_{a b c}^{a}=R_{b c}-R_{c b a}^{a}+0=R_{b c}-R_{c b},
\]
o que nos diz que o tensor de Ricci também é simétrico. Como $R_{ab}$ é simétrico, $\tensor{R}{^a _b}=\tensor{R}{_b ^a}$ e podemos denotar ambos por $R^a_b$. Mais uma contração nos dá o \textit{escalar de curvatura}
\begin{equation}\label{eq:EscalarCurvaturaDefinicao}
	\boxed{
		R \equiv g^{ab}R_{ab}=R^a_a.
	}
\end{equation}

Um último tensor importante é o \textit{tensor de Einstein}, definido por
\begin{equation}\label{eq:TensorEinstein}
	\boxed{
		G_{ab} \equiv R_{ab} - \tfrac{1}{2}Rg_{ab} .
	}
\end{equation}
% {\color{red} falta mostrar que o tensor de Einstein tem divergencia nula}
Em virtude da simetria de todos os termos do lado direito, ele também é simétrico, de modo que possui apenas uma divergência $\tensor{G}{^a^b_{;a}}$. A razão para a importância do tensor de Einstein é que sua divergência é zero. Contraindo-se $a$ com $d$ na identidade de Bianchi \eqref{eq:IdentidadeBianchi} nos dá
\[
	R_{bc;e}+\tensor{R}{^a_{bae;c}}+\tensor{R}{^a_{bec;a}}=0 ,
\]
ou, ao utilizar a equação \eqref{eq:RSimetriaB},
\[
	R_{bc;e}-R_{be;c} + \tensor{R}{^a_{bec;a}} = 0.
\]
Levantando $b$ e contraindo com $e$, temos
\[
	R^b_{c;b}-R_{;c}+\tensor{R}{^a^b_{bc;a}}=0,
\]
mas, pela equação \eqref{eq:RSimetriaC},
\[
	\tensor{R}{^a^b_{bc;a}}=\tensor{R}{^b^a_{cb;a}}=R^a_{c;a}=R^b_{c;b},
\]
reduzindo para $2R^b_{c;b}-R_{;c}=0$, que nos dá $(R^b_c-\tfrac{1}{2}R\delta^b_c)_{;b}=0$ ao dividir-se por dois e usando a última das equações \eqref{eq:DerivadaCovarianteMetrica}. Assim, temos $G^b_{c;b}=0\Rightarrow \tensor{G}{^b^c_{;b}}=0$, como afirmado.

\section{Curvatura e transporte paralelo}\label{sec:CurvaturaTransporteParalelo}

Discutiu-se na Seção \ref{sec:TransporteParalelo} que o transporte paralelo em uma variedade curva dependia do caminho, mostrando especificamente que esse era o caso para uma esfera. No entanto, nós agora formalizamos o conceito de curvatura em termos do tensor de Riemann. Nesta seção, o intuito é deixar clara a conexão entre o transporte paralelo e o tensor de curvatura. Mais especificamente, mostraremos como a mudança $\Delta\lambda^a$ que resulta do transporte paralelo de um vetor $\lambda^a$ ao longo de uma pequena curva fechada ao retor de um ponto $P$ se relaciona com o tensor de curvatora em $P$.

Suponha, por exemplo, que $\lambda^a$ é transportado paralelamente ao longo de uma curva $\gamma$ de um ponto inicial $O$ em que suas componentes são $\lambda_0^a$. Se $\gamma$ é parametrizado por $t$, então $\lambda^a$ satisfaz a equação diferencial
\begin{equation}\label{eq:TransporteParaleloDiferencial}
	\frac{d\lambda^a}{dt} = -\Gamma^a_{bc}\lambda^b\frac{dx^c}{dt},
\end{equation}
de onde podemos chegar que $\lambda$ satisfaz a equação integral
\begin{equation}\label{eq:TransporteParaleloIntegral}
	\lambda^a = \lambda^a_0 - \int\Gamma^a_{bc}\lambda^b dx^c,
\end{equation}
onde a integral é tomada ao longo de $\gamma$ do ponto inicial $O$. Assim, podemos calcular a mudança $\Delta\lambda^a$ à medida que $\lambda$ é transportado ao longo de uma curva pequena próxima a $P$. Se $P$ possui coordenadas $x^a_P$, então os pontos na curva possuirão coordenadas $x^a$ dadas por
\[
	x^a=x^a_P+\xi^a
\]
onde os $\xi^a$ são pequenos. As diferenças entre coordenadas $\xi^a$ podem ser pensadas como um vetor partindo de $P$ para um ponto qualquer de $\gamma$. (veja a Figura \ref{fig:SmallLoop}). Como os $x^a_P$ são constantes, a equação \eqref{eq:TransporteParaleloIntegral} pode ser escrita como
\begin{equation}\label{eq:TransporteParaleloIntegralXi}
	\lambda^a=\lambda^a_0 - \int\Gamma^a_{bc}\lambda^b d\xi^c .
\end{equation}

\begin{figure}[t]
	\centering
	\plot{0.4\linewidth}{figuras/SmallLoop}
	\caption{Uma curva pequena ao redor de $P$.}
	\label{fig:SmallLoop}
\end{figure}

Apesar de termos chegado em uma relação entre $\lambda^a$ e $\lambda^a_0$, nós não podemos calculá-la de uma maneira direta, uma vez que o vetor transportado também aparece na integral à direita. No entanto, podemos utilizar essa relação para conseguir aproximações cada vez melhores que são válidas quando $\lambda^a$ não muda muito de seu valor inicial $\lambda^a_0$, que será o caso para nossa curva pequena próxima de $P$.

Como uma primeira aproximação, podemos utilizar $\lambda^b = \lambda^b_0$ na integral ao lado direito e, então, podemos utilizar o resultado como uma aproximação melhor.
\begin{align*}
	\lambda^a &= \lambda^a_0 - \int\Gamma^a_{bc}\lambda^b_0 d\xi^c \\
	&= \lambda^a_0 - \lambda^b_0\int\Gamma^a_{bc}d\xi^c .
\end{align*}
Utilizando esse resultado ao lado direito, podemos chegar em uma aproximação ainda melhor:
\begin{equation}\label{eq:AproximacaoSegundaOrdemLambda}
\begin{aligned} \lambda^{a} &=\lambda_{0}^{a}-\int \Gamma_{b c}^{a}\left(\lambda_{0}^{b}-\lambda_{0}^{d} \int \Gamma_{d e}^{b} d \xi^{e}\right) d \xi^{c} \\ &=\lambda_{0}^{a}-\lambda_{0}^{b} \int \Gamma_{b c}^{a} d \xi^{c}+\lambda_{0}^{d} \int \Gamma_{b c}^{a}\left(\int \Gamma_{d e}^{b} d \xi^{e}\right) d \xi^{c}. \end{aligned}
\end{equation}
Esse processo pode ser repetido indefinidamente, mas a aproximação dada por \eqref{eq:AproximacaoSegundaOrdemLambda} é o suficiente para o nosso propósito, o qual envolve utilizar aproximações de segunda ordem em $\xi^a$.

Na integral do segundo termo da direita, podemos usar a aproximação de primeira ordem $\Gamma^a_{bc}=(\Gamma^a_{bc})_P+(\partial_d\Gamma^a_{bc})_P\xi^d,$ uma vez que será integrado com respeito a $\xi^c$, resultando em uma aproximação de segunda ordem. Para a integral repetida no terceiro termo, podemos aproximar $\Gamma^a_{bc}$ por $(\Gamma^a_{bc})_P$, já que será integrado duas vezes. Usando essas aproximações para integrar de $O$ ao longo de $\gamma$ de volta a $O$, temos
\[
	\oint\Gamma^a_{bc}d\xi^c=(\Gamma^a_{bc})_P\oint d\xi^c+(\partial_d\Gamma^a_{bc})_P\oint\xi^d d\xi^c = (\partial_d \Gamma^a_{bc})_P\oint\xi^d d\xi^c
\]
(como $\oint d\xi^c=0$) e
\[
	\oint\Gamma^a_{bc}\left( \int\Gamma^b_{de} d\xi^e\right)d\xi^c = (\Gamma^a_{bc}\Gamma^b_{de})_P\oint \left( \int d\xi^e \right)d\xi^c = (\Gamma^a_{bc}\Gamma^b_{de})_P \oint\xi^e d\xi^c ,
\]
então, da equação \eqref{eq:AproximacaoSegundaOrdemLambda}, a variação em $\lambda^a$ ao transportá-lo em $\gamma$ é
\[
	\Delta\lambda^a = -\lambda^b_0(\partial_d\Gamma^a_{bc})_P\oint\xi^d d\xi^c+\lambda^d_0(\Gamma^a_{bc}\Gamma^b_{de})_P\oint\xi^e d\xi^c,
\]
que se reduz para
\begin{equation}\label{eq:VariacaoLambda}
	\Delta\lambda^a=-\lambda_0^b(\partial_c\Gamma^a_{bc}-\Gamma^a_{ed}\Gamma^e_{bc})_P\oint\xi d\xi^d ,
\end{equation}
ao fazer uma troca de índices. Como $\oint d(\xi^c\xi^d)=0$, temos que
\[
	f^{cd}\equiv\oint \xi^c d\xi^d = \tfrac{1}{2}\oint(\xi^c d\xi^d-\xi^d d\xi^c),
\]
que é antissimétrico. Assim, a partir da equação \eqref{eq:VariacaoLambda}
\begin{align*}
\Delta\lambda^a &=-\lambda_0^b(\partial_c\Gamma^a_{bc}-\Gamma^a_{ed}\Gamma^e_{bc})_P f^{cd} \\
&=-\lambda_0^b(\partial_d\Gamma^a_{bc}-\Gamma^a_{ec}\Gamma^e_{bd})_P(-f^{cd}) \qquad \text{(trocando $c$ por $d$)}.
\end{align*}
Somando e dividindo por 2, temos
\[
\Delta\lambda^a = -\tfrac{1}{2}(
	\partial_c\Gamma^a_{bd}-\partial_d\Gamma^a_{bc}+\Gamma^a_{ec}\Gamma^e_{bd}-\Gamma^a_{ed}\Gamma^e_{bc}
	)_P\lambda^b_0 f^{cd} .
\]
Isto é, 
\begin{equation}\label{eq:VariacaoLambdaRiemann}
	\boxed{
		\Delta\lambda^a=-\tfrac{1}{2}(\tensor{R}{^a_{bcd}})_P\lambda^b_0 f^{cd}.
	}
\end{equation}
Essa equação estabelece a relação básica entre o tensor de curvatura em um ponto $P$ e o transporte paralelo ao redor de um ciclo pequeno próximo a $P$.

\section{Desvio geodésico}\label{sec:DesvioGeodesico}
Outro contexto em que a curvatura aparece é na equação do desvio geodésico, que vamos deduzir nesta seção.

Considere duas geodésicas próximas, $\gamma$ dada por $x^a(u)$ e $\tilde{\gamma}$, dada por $\tilde{x}^a(u)$, ambas com parâmetros afins. Seja também $\xi^a(u)$ o ``vetor'' pequeno que conecta os pontos com mesmos parâmetros, i.e, $\xi^a(u)=\tilde{x}^a(u)-x^a(u)$ (veja a Figura \ref{fig:DesvioGeodesico}). Se nenhuma geodésica é nula, o comprimento de arco $s$ pode ser utilizado.

\begin{figure}[th]
	\centering
	\plot{0.4\linewidth}{figuras/DesvioGeodesico}
	\caption{Duas geodésicas próximas.}
	\label{fig:DesvioGeodesico}
\end{figure}

Já que $\gamma$ e $\tilde{\gamma}$ são geodésicas, temos
\begin{equation}\label{eq:GeodesicaTil}
	\frac{d^2\tilde{x}^a}{du^2}+\tilde{\Gamma}^a_{bc}\frac{d\tilde{x}^b}{du}\frac{d\tilde{x}^c}{du}=0
\end{equation}
e
\begin{equation}\label{eq:GeodesicaSemTil}
	\frac{d^2x^a}{du^2}+{\Gamma}^a_{bc}\frac{dx^b}{du}\frac{dx^c}{du}=0.
\end{equation}
Mas, em primeira ordem,
\[
	\tilde{\Gamma^a_{bc}}=\Gamma^a_{bc}+\Gamma^a_{bc,d}\xi^d,
\]
e subtrair a equação \eqref{eq:GeodesicaSemTil} de \eqref{eq:GeodesicaTil} nos dá
\[
	\ddot{\xi}^a+\Gamma^a_{bc,d}\dot{x}^b\dot{x}^c\xi^d+\Gamma^a_{bc}\dot{x}^b\dot{\xi}^c+\Gamma^a_{bc}\dot{\xi}^b\dot{x}^c=0,
\]
onde os pontos denotam as derivadas com respeito a $u$ e apenas os termos de primeira ordem foram considerados. Isso pode ser escrito como
\[
	d(\dot{\xi}^a+  \Gamma^a_{bc}\xi^b\dot{x}^c)/du -
	\Gamma^a_{bc,d}\xi^b\dot{x}^c\dot{x}^d -
	\Gamma^a_{bc}\xi^b\ddot{x}^c +
	\Gamma^a_{bc,d}\dot{x}^b\dot{x}^c\xi^d +
	\Gamma^a_{bc}\dot{x}^b\dot{\xi}^c = 0 .
\]
Substituindo $\ddot{x}^c$ da equação \eqref{eq:GeodesicaSemTil}, podemos rearranjar para
\[
	d(\dot{\xi}^a+  \Gamma^a_{bc}\xi^b\dot{x}^c)/du +
	\Gamma^a_{de}(\dot{\xi}^d+\Gamma^d_{bc}\xi^b\dot{x}^c)\dot{x}^e -
	\Gamma^a_{de}\Gamma^d_{bc}\xi^b\dot{x}^c\dot{x}^e 
\]
\[
	- \Gamma^a_{bc,d}\xi^b\dot{x}^c\dot{x}^d+
	\Gamma^a_{bc}\xi^b\Gamma^c_{de}\dot{x}^d\dot{x}^e +
	\Gamma^a_{bc,d}\dot{x}^b\dot{x}^c\xi^d = 0 .
\]
Ao renomear sufixos, podemos reduzir essa expressão para
\[
	D^2\xi^a/du^2+(
		\Gamma^a_{cd,b} - \Gamma^a_{bc,d}+\Gamma^a_{be}\Gamma^e_{dc}-\Gamma^a_{ed}\Gamma^e_{bc}
		)\xi^b\dot{x}^c\dot{x}^d = 0 , 
\]
que, por sua vez, pode ser escrito de maneira mais compacta como
\begin{equation}\label{eq:DesvioGeodesico}
	\boxed{
	D^2\xi^a/du^2 + R^a_{cbd}\xi^b\dot{x}^c\dot{x}^d=0.
	}
\end{equation}
Essa é a \textit{equação do desvio geodésico}. 

Em uma variedade plana, $\tensor{R}{^a_{bcd}}=0$ e, em coordenadas cartesianas, $D/du=d/du$, de modo que a equação \eqref{eq:DesvioGeodesico} é reduzida a $d^2\xi^a/du^2=0$, que implica que $\xi^a(u)=A^a u+B^a$, onde $A^a,B^a$ são constantes. Assim, o vetor de separação aumenta linearmente com $u$ (e com $s$ se $\gamma$ for não-nula). No entanto, em uma variedade curva, $\tensor{R}{^a_{bcd}}\neq0$ e não temos essa relação linear.

\section{Equações de campo de Einstein}\label{sec:EquacoesCampoEinstein}

O tensor de métrica contém dois tipos de informação distintos:
\begin{enumerate}[label=\textbf{(\roman*)}]
	\item A informação com respeito ao sistema de coordenadas utilizado (coordenadas esféricas, cartesianas etc);
	\item A informação (mais importante) com respeito a existência de potenciais gravitacionais.
\end{enumerate}

Vimos na seção \ref{sec:PotencialGravitacionalGeodesica} que em um sistema quase plano, $g_{00}$ é basicamente o potencial newtoniano. Em um sistema de coordenadas mais geral, esse potencial estaria difuso ao longo de $g_{\mu\nu}$, de modo que todas as componentes podem ser tratadas como potenciais gravitacionais.

Como vimos na seção \ref{sec:EnergiaMomentoFluidos}, o conteúdo de matéria contido no espaço-tempo é resumido no tensor de energia-momento $T^{\mu\nu}$. Assim, se a matéria causa a geometria, poderíamos tentar postular
\[
	g^{\mu\nu}=\kappa T^{\mu\nu},
\]
onde $\kappa$ é uma constante de acoplamento. Isso a princípio parece plausível, pois ambos $g^{\mu\nu}$ e $T^{\mu\nu}$ são simétricos e possuem divergência nula (equação \eqref{eq:DerivadaCovarianteMetrica}). No entanto, essa equação não se reduz à equação de Poisson $\nabla^2V=4\pi G\rho$ no limite newtoniano. Como $g_{\mu\nu}$ são os potenciais gravitacionais, precisamos de um tensor simétrico envolvendo as derivadas de segunda ordem de $g^{\mu\nu}$ na equação.

O tensor de Ricci é simétrico e possui as segundas derivadas de $g_{\mu\nu}$. Poderíamos, então, tentar novamente postular uma equação do tipo $R^{\mu\nu}= \kappa T^{\mu\nu}$. No entanto, $R^{\mu\nu}$ não satisfaz $\tensor{R}{^{\mu\nu}_{;\mu}}=0$. Assim, modificando a equação para
\begin{equation}\label{eq:EquacoesdeCampo}
	\boxed{
		R^{\mu\nu}-\tfrac{1}{2}Rg^{\mu\nu}=\kappa T^{\mu\nu} ,
	}
\end{equation}
temos que o lado esquerdo é o tensor de Einstein $G^{\mu\nu}$, que vimos na seção \ref{sec:TensorCurvatura}, que satisfaz $\tensor{G}{^{\mu\nu}_{;\mu}}=0$, de modo que a equação \eqref{eq:EquacoesdeCampo} parece satisfatória em todos os aspectos. Note que temos agora dez equações de campo, em vez de uma única, como na teoria newtoniana. Uma forma alternativa para escrevê-las é
\begin{equation}\label{eq:EquacoesdeCampoAlternativa}
	R^{\mu\nu}=\kappa(T^{\mu\nu}-\tfrac{1}{2}Tg^{\mu\nu}) ,
\end{equation}
onde $T \equiv T^\mu_\mu$.

Lembre-se que $T^{\mu\nu}$ contém toda forma de energia e momento. Por exemplo, se há radiação eletromagnética presente, então isso deve estar incluso em $T^{\mu\nu}$. Uma região no espaço-tempo em que $T^{\mu\nu}=0$ é chamada de \textit{vazia}, sendo desprovida de matéria, energia de radiação e momento. Pode-se ver da equação \eqref{eq:EquacoesdeCampoAlternativa} que as equações de campo para o espaço-tempo vazio são
\begin{equation}\label{eq:EquacoesCampoVazio}
	R^{\mu\nu}=0.
\end{equation}

A equação de Poisson pode ser recuperada das equações de campo ao considerarmos sua componente $00$ na aproximação de campo fraco. Utilizando a versão covariante da equação \eqref{eq:EquacoesdeCampoAlternativa} para essa componente, temos
\begin{equation}\label{eq:EquacaoCampo00}
	R_{00}=\kappa(T_{00}-\tfrac{1}{2}Tg_{00}).
\end{equation}
Vamos utilizar um sistema de coordenadas como na seção \ref{sec:PotencialGravitacionalGeodesica}, onde $g_{\mu\nu}=\eta_{\mu\nu}+h_{\mu\nu}$, e os produtos de $h_{\mu\nu}$ podem ser desprezados. Também assumimos a condição quasi-estática utilizada naquela seção.

Vamos assumir que o nosso campo gravitacional fraco provém de um fluido perfeito cujas partículas têm (no nosso sistema de coordenadas) velocidades $v$ muito menores do que $c$, de modo que $\gamma = (1-v^2/c^2)^{-1/2}\approx 1$. Vamos utilizar também que $p/c^2 \ll \rho$, que é válido para a maioria das distribuições clássicas, de modo que o tensor de energia-momento é
\[
	T_{\mu\nu}=\rho u_\mu u_\nu .
\]
Isso nos dá $T=\rho c^2$ e a equação \ref{eq:EquacaoCampo00} se torna
\[
	R_{00}=\kappa\rho(u_0u_o-\tfrac{1}{2}c^2g_{00}),
\]
mas $u_0 \approx c$ e $g_{00}\approx 1$, de modo que temos
\begin{equation}\label{eq:Ricci00KappaRhoC}
	R_{00}\approx\tfrac{1}{2}\kappa\rho c^2,
\end{equation}
onde, da equação \eqref{eq:DefinicaoTensorRiemann}, temos 
\begin{equation*}
	R_{00}=\partial_0\Gamma^\mu_{0\mu}-\partial_\mu\Gamma^\mu_{00}+\Gamma^\nu_{0\mu}\Gamma^\mu_{\nu0}-\Gamma^\nu_{00}\Gamma^\mu_{\nu\mu.
	} .
\end{equation*}

No nosso sistema de coordenadas, temos que $\Gamma^\mu_{\nu\sigma}$ são pequenos, e portanto podemos ignorar os dois ultimos termos. Utilizando a condição quasi-estática, temos
\[
	R_{00}\approx-\partial_i\Gamma^i_{00} .
\]
Da seção \ref{sec:PotencialGravitacionalGeodesica}, temos que nessa aproximação vale
\[
	\Gamma^i_{00}=\tfrac{1}{2}\delta^{ij}\partial_j h_{00},
\]
reduzindo a equação \eqref{eq:Ricci00KappaRhoC} a
\[
	-\tfrac{1}{2}\delta^{ij}\partial_j\partial_j h_{00}\approx\tfrac{1}{2}\kappa\rho c^2.
\]
Mas $\delta^{ij}\partial_j\partial_j=\nabla^2$ e, da equação \eqref{eq:RelacaoG00Potencial}, $h_{00}=2V/c^2$, onde $V$ é o potencial gravitacional. Isso resulta em 
\[
	\nabla^2V\approx-\tfrac{1}{2}\kappa\rho c^4,
\]
que corresponde à equação de Poisson, desde que identifiquemos a constante $\kappa$ na equação de Einstein com $-8\pi G/c^4$. Assim, a equação se torna
\begin{equation}
	\boxed{
		\nabla^2V\approx 4\pi G\rho
	}
\end{equation}

\section*{Conclusões}

Neste relatório, conforme previsto pelo projeto proposto, fez-se primeiramente uma revisão de Relatividade Especial, no capítulo \ref{cap:RelatividadeEspecial}, onde chegamos nas transformações de Lorentz, comparamos a geometria de RE com a geometria hiperbólica e discutimos a mecânica relativística.

Após isso, começou-se o estudo da Teoria da Relatividade Geral, firmando a base matemática, passando pelos tópicos de sistemas de coordenadas no espaço euclidiano e suas transformações, variedades e campos tensoriais, incluindo o tensor de métrica (capítulo \ref{cap:CamposVetoriaisETensoriais}). Após isso, tratamos de derivadas e geodésicas no espaço curvo, bem como conceitos relacionados tais como o transporte paralelo e estudamos o espaço-tempo da Relatividade Geral, de modo a retomar algumas leis clássicas, como a Lei da Gravitação de Newton para campos gravitacionais fracos (capítulo \ref{cap:EspacoTempoRelatividadeGeralTrajetoriaParticulas}).

Definimos, também, os tensores de Riemann, Ricci e, fazendo a introdução de conceitos físicos, culminamos na definição do tensor de energia-momento no contexto de dinâmica de fluidos. Por fim, chegamos nas equações de Einstein, relacionando conceitos matemáticos (tensor de curvatura de Einstein) com conceitos físicos (tensor de energia-momento), no capítulo \ref{cap:EquacoesDeCampoCurvatura}.

Tudo está de acordo com o projeto inicial e pretendemos continuar seguindo o cronograma. Isto é, o espaço-tempo de Schwarzschild será análisado, cuja métrica será deduzida a partir das equações de Einstein. Com a métrica obtida, serão estudadas as propriedades físicas desse espaço-tempo como simetrias, geodésicas de partículas massivas e da luz, potencial efetivo e efeitos de \textit{redshift}. Por fim, será descrito o buraco negro de Schwarzschild, identificando propriedades físicas importantes, como o horizonte de eventos, estrutura causal e diagrama de Carter-Penrose.

