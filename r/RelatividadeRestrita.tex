\chapter{Relatividade Restrita}

\section{*Introdução}

{\color{red} Pensar em como introduzir o texto.}

\section{*Postulados da Relatividade Restrita}




A Relatividade Restrita baseia-se em dois postulados:

\begin{enumerate}[label=\textbf{(\Roman*)}]
    \item As leis da física se aplicam a todos os referenciais inerciais \label{post:leis}
    \item A velocidade da luz é a mesma para todos os referenciais inerciais. \label{post:velocidade}
\end{enumerate}

O primeiro postulado nós dá uma noção de referenciais preferenciais. Um referencial inercial é um em que um objeto inicialmente em repouso continuará em repouso. Por causa da gravidade, referenciais inerciais devem estar em queda livre, mas a Relatividade Restrita descreve eventos sem gravidade, de modo que, na prática, podemos descrever referenciais inerciais em termos de seu movimento relativo em velocidade constante.

O postulado \ref{post:leis} é uma generalização do princípio da \textit{Invariância de Galileu}, de modo a abranger a eletrodinâmica, além da mecânica. No entanto, as equações de Maxwell referenciam explicitamente a velocidade da luz. De fato, ela é escrita em termos da permissividade elétrica $\epsilon_0$ e a permeabilidade magnética $\mu_0$ do vácuo, que são constantes que podem ser medidas experimentalmente. As equações de Maxwell prevêm que as ondas eletromagnéticas (incluindo a luz) possuem velocidade (no vácuo) de 
\[
    c = \frac{1}{\sqrt{\epsilon_0\mu_0}}
\]
No entanto, não há uma referência a qual referencial essa velocidade é observada. O experimento de Michelson-Morley, realizado em 1887, foi realizado a fim de mostrar que essa velocidade é em relação ao éter, de modo que nós poderíamos medir nosso próprio movimento em relação ao éter ao medirem-se as variações de $c$ dependentes da direção. No entanto, o experimento mostrou que não há tais variações. Einstein então argumentou que não existe, portanto, éter. O postulado \ref{post:leis} junto com as equações de Maxwell então, nos leva ao postulado \ref{post:velocidade}.

Uma conclusão imediata que segue de ambos os postulados é que dois observadores podem não concordar na simultaneidade entre dois eventos -- na verdade, eles discordam, no geral. Por exemplo, considere um trem com uma lâmpada no centro (Figura \ref{fig:Trens} \subref{fig:trem})

Para um observador dentro do trem, independentemente do movimento deste, os dois feixes de luz chegarão às paredes ao mesmo tempo. No entanto, para um observador que observe o trem movendo-se para a direita, o feixe viajando para a equerda chegará primeiro (Figura \ref{fig:Trens} \subref{fig:TremMovimento}). Para esse observador, o trem possui uma velocidade adicional $v$, mas a luz continua com a mesma velocidade $c$ (ambos os feixes), segundo o postulado \ref{post:velocidade}, resultando em distâncias diferentes para serem percorridas até as paredes.


\begin{figure*}[t]
    \centering
    \begin{subfigure}[t]{0.5\textwidth}
        \centering
        \plot{0.83\linewidth}{figuras/trem}
        \caption{}
        \label{fig:trem}
    \end{subfigure}%
    ~ 
    \begin{subfigure}[t]{0.5\textwidth}
        \centering
        \plot{0.9\linewidth}{figuras/TremMovimento}
        \caption{}
        \label{fig:TremMovimento}
    \end{subfigure}
    \caption{Feixes de luz partindo do centro de um trem ao mesmo tempo para (a) um observador que anda junto ao trem e (b) para um observador que vê o trem movendo-se com velocidade $v$ para a direita.}
\label{fig:Trens}
\end{figure*}



 
\begin{figure}[th]
    \centering
    \plot{0.4\linewidth}{figuras/doppler}
    \caption{Efeito Doppler Relativístico.}
    \label{fig:doppler}
\end{figure}
