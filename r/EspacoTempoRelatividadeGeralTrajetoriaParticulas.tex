\chapter{O espaço-tempo da relatividade geral e trajetórias de partículas}\label{cap:EspacoTempoRelatividadeGeralTrajetoriaParticulas}
\section{*Geodésicas}\label{sec:Geodesicas}
Uma geodésica num espaço Euclideano é simplesmente uma reta, que pode ser caracterizada como a menor curva que liga dois pontos. Pode-se generalizar essa definição para a geodésica de uma variedade, onde o tensor de métrica 
%campo tensorial de métrica?
dá o comprimento de uma curva pela integral da equação \textbf{eq:ComprimentoCurvaTensorMetrica}. No entanto, essa abordagem pode trazer problemas quando podem existir curvas (ou partes delas) com comprimento nulo. Dessa forma, caracterizaremos uma reta por sua \textit{retidão} e usaremos isso para definir geodésicas numa variedade.

Utilizando o comprimento de arco $ s $ medido de um ponto de base numa reta como parâmetro, então os vetores tangentes $ \boldsymbol{\lambda} \equiv \dot{\mathbf{r}}(s) $ tem módulo constante
 -- dado que são vetores unitários --%Exercício 1.3.3. Foster
, então, dizer que têm direção constante é o mesmo que dizer

\begin{equation}\label{eq:VetorTangenteConstante}
d \boldsymbol{\lambda} / d s=\mathbf{0}
\end{equation}

Utilizando essa equação em uma base arbitraria de coordenadas $ u^i $ e sua base natural $ \left\{\mathbf{e}_{i}\right\} $, onde $ \boldsymbol{\lambda}=\lambda^{i} \mathbf{e}_{i} $ e os pontos indicam derivadas em relação a $ s $, temos

\begin{equation}\label{eq:VetorTangenteConstanteBase}
0=d \boldsymbol{\lambda} / d s=d\left(\lambda^{i} \mathbf{e}_{i}\right) / d s=\dot{\lambda}^{i} \mathbf{e}_{i}+\lambda^{i} \dot{\mathbf{e}}_{i}
\end{equation}

Como $ \dot{\mathbf{e}}_{i}=\partial_{j} \mathbf{e}_{i} \dot{u}^{j} $ e o podemos escrever $ \partial_{j} \mathbf{e}_{i}  $ em termos da base $ \left\{\mathbf{e}_{i}\right\} $, de modo que

\begin{equation}\label{eq:GammaDefinicao}
\partial_{j} \mathbf{e}_{i}=\Gamma_{i j}^{k} \mathbf{e}_{k}
\end{equation}

O que dá origem a 27 quantidades $ \Gamma_{i j}^{k} $ definidas em cada ponto no espaço. Juntando as equações \eqref{eq:GammaDefinicao} e \eqref{eq:VetorTangenteConstanteBase}, chegamos em
\begin{equation}\label{eq:CoordenadasLambdaE}
	\left(\dot{\lambda}^{i}+\Gamma_{j k}^{i} \lambda^{j} \dot{u}^{k}\right) \mathbf{e}_{i}=\mathbf{0}
\end{equation}

Uma vez que $ \lambda^{i}=\dot{u}^{i}=d u^{i} / d s $, chegamos que as componentes $ d u^{i} / d s $ do vetor tangente à reta satisfazem

\begin{equation}\label{eq:ComponentesUGamma}
\frac{d^{2} u^{i}}{d s^{2}}+\Gamma_{j k}^{i} \frac{d u^{j}}{d s} \frac{d u^{k}}{d s}=0
\end{equation} 

A fim de encontrar $ \Gamma_{j k}^{i} $ em termos de quantidades conhecidas, primeiramente notamos que

\[
\partial_{j} \mathbf{e}_{i}=\frac{\partial^{2} \mathbf{r}}{\partial u^{j} \partial u^{i}}=\frac{\partial^{2} \mathbf{r}}{\partial u^{i} \partial u^{j}}=\partial_{i} \mathbf{e}_{j}
\]

de modo que $ \Gamma_{i j}^{k} \mathbf{e}_{k}=\Gamma_{j i}^{k} \mathbf{e}_{k} $. Tomando o produto interno com $ \mathbf{e}^{l} $, temos a propriedade simétrica:


\begin{equation}\label{eq:GammaSimetria}
\boxed{\Gamma_{i j}^{l}=\Gamma_{j i}^{l}}
\end{equation}

Usando que $ g_{i j}=\mathbf{e}_{i} \cdot \mathbf{e}_{j} $, temos

\[
\partial_{k} g_{i j}=\partial_{k} \mathbf{e}_{i} \cdot \mathbf{e}_{j}+\mathbf{e}_{i} \cdot \partial_{k} \mathbf{e}_{j}=\Gamma_{i k}^{m} \mathbf{e}_{m} \cdot \mathbf{e}_{j}+\mathbf{e}_{i} \cdot \Gamma_{j k}^{m} \mathbf{e}_{m}
\]


\begin{equation}\label{eq:deltak}
\Rightarrow \partial_{k} g_{i j}=\Gamma_{i k}^{m} g_{m j}+\Gamma_{j k}^{m} g_{i m}
\end{equation}

Apenas trocando os índices, obtemos

\begin{equation}\label{eq:deltai}
\partial_{i} g_{j k}=\Gamma_{j i}^{m} g_{m k}+\Gamma_{k i}^{m} g_{j m}
\end{equation}

e

\begin{equation}\label{eq:deltaj}
\partial_{j} g_{k i}=\Gamma_{k j}^{m} g_{m i}+\Gamma_{i j}^{m} g_{k m}
\end{equation}

Subtraindo-se a equação \eqref{eq:deltaj} da soma (\eqref{eq:deltai}+\eqref{eq:deltak}), obtemos

\[2 \Gamma_{k_{i}}^{m} g_{m j}=\partial_{k} g_{i j}+\partial_{i} g_{j k}-\partial_{j} g_{k i}\]

que, por sua vez, ao contrair-se com $ \tfrac{1}{2} g^{l j} $ gera

\begin{equation}\label{eq:Gamma}
\boxed{\Gamma_{k i}^{l}=\tfrac{1}{2} g^{l j}\left(\partial_{k} g_{i j}+\partial_{i} g_{j k}-\partial_{j} g_{k i}\right)}
\end{equation}


Assim, a equação \eqref{eq:ComponentesUGamma} com o tensor $ \Gamma_{jk}^i $ dado pela equação \eqref{eq:Gamma} é a \textit{equação geodésica} para um espaço euclidiano.

Para um parâmetro $ t $ da forma $ t = As+B $, com $ A\neq0,B $ constantes, a equação geodésica possui a mesma forma indicada:

\begin{equation}\label{eq:GeodesicaEmT}
\frac{d^{2} u^{i}}{d t^{2}}+\Gamma_{j k}^{i} \frac{d u^{j}}{d t} \frac{d u^{k}}{d t}=0
\end{equation}

Os parâmetros cujas equações são dessa forma são chamados de \textit{parâmetros afins}. Nesses casos, $ ds/dt $ é constante e, se $ t $ for pensado como o tempo, a geodésica é atravessada a velocidade constante.

A equação \eqref{eq:GeodesicaEmT} é um sistema de equações diferenciais de segunda ordem cuja solução geral $ u^i(t) $ gera as geodésicas do espaço euclidiano em qualquer sistemas de coordenadas que estivermos utilizando.

Utilizando esses resultados, podemos definir uma \textit{geodésica afim} como uma variedade Rimanniana ou pseudo-Rimanniana como uma curva $ x^a(u) $ dada satisfazendo

\begin{equation}\label{eq:GeodesicaAfimPseudoRimanniana}
\boxed{\frac{d^{2} x^{a}}{d u^{2}}+\Gamma_{b c}^{a} \frac{d x^{b}}{d u} \frac{d x^{c}}{d u}=0}
\end{equation}

Onde $ \Gamma^{a}_{bc} $ é dado por $ \Gamma_{b c}^{a}=\tfrac{1}{2} g^{a d}\left(\partial_{b} g_{d c}+\partial_{c} g_{b d}-\partial_{d} g_{b c}\right) $. Essas quantidades são chamadas de \textit{coeficientes de conexão}.
Em uma geodésica afim, o módulo do vetor tangente $ \dot{x}^a $ permanece constante e, caso a geodésica não seja nula, o parâmetro afim é dado pelo comprimento de arco $ s $ por $ u=As+B $ onde $ A\neq0,B $ são constantes. Em uma geodésica nula, os vetores tangentes satisfazem $ g_{a b} \dot{x}^{a} \dot{x}^{b}=0 $ e o comprimento de arco não pode ser utilizado como parâmetro.


{\color{red}Escrever sobre Lagrangiana}


 
\section{*Transporte paralelo}\label{sec:TransporteParalelo} %"Vetores paralelos ao longo de uma curva"

Seja $ \gamma $ uma curva no espaço euclidiano tridimensional dado parametricamente por $ u^i(t) $ e seja $ P_0 $ com parâmetro $ t_0 $ ser o ponto inicial dessa curva onde é dado um vetor $ \boldsymbol{\lambda}_0 $. Pode-se pensar em transportar $ \boldsymbol{\lambda}_0 $ ao longo de $ \gamma $ sem nenhuma mudança em seu comprimento ou direção, de modo a obter um vetor paralelo $ \boldsymbol{\lambda}(t) $ em cada ponto da curva. O resultado é um campo vetorial paralelo ao longo de $ \gamma $ gerado pelo \textit{transporte paralelo} de $ \boldsymbol{\lambda}_0 $. Como não há mudanças em sua direção e sentido, o vetor satisfaz a equação diferencial 

\begin{equation}\label{eq:TransporteParalelo}
d \boldsymbol{\lambda} / d t=\mathbf{0}
\end{equation}

que é idêntica à equação \eqref{eq:VetorTangenteConstante}. Assim, fazendo as mesmas modificações e, à partir da equação \eqref{eq:CoordenadasLambdaE}, chegamos que as componentes do vetor satisfazem 

\begin{equation}\label{eq:CoordenadasLambda}
\dot{\lambda}^{i}+\Gamma_{j k}^{i} \lambda^{j} \dot{u}^{k}=0
\end{equation} 

em que os coeficientes de conexão são dados pela equação \eqref{eq:Gamma}.

\begin{exemplo}
Considere uma esfera de raio $a$, com coordenadas $u^1\equiv \theta,\, u^2\equiv\phi$, onde $\theta,\phi$ são os ângulos polares usuais das coordenadas esféricas, com $0\leq\theta\leq\pi$ e $0\leq\phi\leq 2\pi$.

Então, 
\[
\left[g_{A B}\right]=\left[\begin{array}{cc}{a^{2}} & {0} \\ {0} & {a^{2} \sin ^{2} \theta}\end{array}\right]
\]

e seus únicos coeficientes de conexão não nulos são {\color{red}(FAZER EXERCÍCIO 2.1.5 E CITAR AQUI)} 

\[
	\Gamma_{22}^{1}=-\sin \theta \cos \theta, \quad \Gamma_{12}^{2}=\Gamma_{21}^{2}=\cot \theta
\]

\begin{figure}[th]
	\centering
	\plot{0.4\linewidth}{figuras/EsferaTransporteParalelo}
	\caption{Transporte paralelo ao longo de uma circunferência de latitude $\gamma$}
	\label{fig:EsferaTransporteParalelo}
\end{figure}



Se transportarmos um vetor $\boldsymbol{\lambda}$ paralelamente ao longo do círculo de latitude $\gamma$ dado por $\theta=\theta_0$, começando e terminando no ponto $P_0$ onde $\phi=0$ (veja figura \ref{fig:EsferaTransporteParalelo}). O círculo é dado parametricamente por
\[u^{A}(t)=\theta_{0} \delta_{1}^{A}+t \delta_{2}^{A}, \quad 0 \leq t \leq 2 \pi,\]
de modo que $\dot{u}^{A}=\delta_{2}^{A}$ e a equação de transporte paralelo {\color{red}(citar?)} se torna $\dot{\lambda}^{A}+\Gamma_{B 2}^{A} \lambda^{B}=0$, que é equivalente ao par
\begin{equation}\label{eq:EquacoesTransporteLambda}
\left\{\begin{aligned} \dot{\lambda}^{1}-\sin \theta_{0} \cos \theta_{0} \lambda^{2} &=0 \\ \dot{\lambda}^{2}+\cot \theta_{0} \lambda^{1} &=0 \end{aligned}\right. .
\end{equation}

Considere que $\boldsymbol{\lambda}$ inicialmente seja um vetor unitário e faça um ângulo $\alpha$ ao leste do sul. Assim,
\begin{equation}\label{eq:CondicoesIniciaisLambda}
	\lambda^{1}(0)=a^{-1} \cos \alpha, \quad \lambda^{2}(0)=\left(a \sin \theta_{0}\right)^{-1} \sin \alpha ,
\end{equation}
como pode-se perceber pelo fato de que
\[
	g_{AB}\lambda^A(0)\lambda^B(0)=(a^2)(a^{-1}\cos\alpha)^2+(a^2\sin^2\theta)(a\sin\theta)^{-2}\sin^2\alpha=\cos^2\alpha+\sin^2\alpha=1
\]
e
\[
	g_{AB}\lambda^A(0)S^B = g_{AB}\lambda^A(0)a^{-1}\delta^B_1=g_{A1}\lambda^A(0)a^{-1}=a^2(a^{-1}\cos\alpha)a^{-1}=\cos\alpha
\]
onde $S^A\equiv a^{-1}\delta^A_1$ é o vetor unitário que aponta para o sul em $P_0$

Assim, temos um problema de valor inicial consistindo nas equações \eqref{eq:EquacoesTransporteLambda} com condições iniciais \eqref{eq:CondicoesIniciaisLambda}. Substituindo a primeira equação de \eqref{eq:EquacoesTransporteLambda} na derivada da segunda, chegamos em
\[
\ddot{\lambda}^2+\cot\theta_0(\sin\theta_0\cos\theta_0\lambda^2)=0	
\]
\[
\Rightarrow \ddot{\lambda}^2+\omega^2\lambda^2=0, \qq{onde} \omega=\cos\theta_0,
\]
cuja solução é da forma $\lambda^2(t)=A\cos(\omega t+\varphi)$. Derivando e substituindo na segunda equação de \eqref{eq:EquacoesTransporteLambda}, temos
\[
\lambda^1=-\tan\theta_0\dot{\lambda}^2=-\tan\theta_0(-\omega A\sin(\omega t+\varphi))	= A\sin\theta_0\sin(\omega t+\varphi).
\]
Aplicando-se as condições iniciais \eqref{eq:CondicoesIniciaisLambda}:
\[
\left\{\begin{array}{l}
{A\sin\theta_0\sin\varphi=a^{-1}\cos\alpha} \\ 
{A\cos\varphi=(a\sin\theta_0)^{-1}\sin\alpha}
\end{array}\right.	
\]
chegamos nas soluções $\varphi=\pi/2-\alpha$ e $A=(a\sin\theta_0)^{-1}$. Substituindo na solução e utilizando $\cos(\pi/2-x)=\sin(x)$, chegamos na solução:
\begin{equation}\label{eq:SolucaoTransporteLambda}
	\left\{\begin{array}{l}
	{\lambda^{1}=a^{-1} \cos (\alpha-\omega t)} \\ 
	{\lambda^{2}=\left(a \sin \theta_{0}\right)^{-1} \sin (\alpha-\omega t)}
	\end{array}\right.
\end{equation}

Ao completar-se o caminho $\gamma$, o vetor obtido por transporte paralelo possui componentes
\[
\left\{\begin{array}{l}{\lambda^{1}(2 \pi)=a^{-1} \cos (\alpha-2 \pi w)} \\ {\lambda^{2}(2 \pi)=\left(a \sin \theta_{0}\right)^{-1} \sin (\alpha-2 \pi \omega)}\end{array}\right.
\]

Pode-se perceber que $g_{A B} \lambda^{A}(2 \pi) \lambda^{B}(2 \pi)=1$, então $\lambda^A(2\pi)$ é um vetor unitário, como deveria ser, mas sua direção muda (a não ser que $\omega=0$, como no equador). Como
\[
\begin{aligned} g_{A B} \lambda^{A}(0) \lambda^{B}(2 \pi) &=\cos \alpha \cos (\alpha-2 \pi \omega)+\sin \alpha \sin (\alpha-2 \pi \omega) \\ &=\cos (\alpha-(\alpha-2 \pi \omega)) \\ &=\cos 2 \pi \omega ,\end{aligned}
\]
temos que o vetor final faz um ângulo $2\pi\omega$ com o inicial.

\end{exemplo}

\section{*Diferenciação absoluta e covariante}\label{sec:DiferenciacaoAbsolutaCovariante}

Considere um campo vetorial $ \lambda^a(u) $ definido ao longo de uma curva $ \gamma $ dado parametricamente por $ x^a(u) $. As $ N $ quantidades $ d \lambda^{a} / d u $ não são componentes de um vetor. Para ver isso, utilizamos um outro sistema de coordenadas e olhamos para as correspondentes quantidades $ d \lambda^{a^{\prime}} / d u $ para ver como elas se relacionam com as originais. Elas são dadas por

\begin{equation}\label{eq:TransformacaoCoordenadasDerivadaCurva}
d \lambda^{a^{\prime}} / d u=d\left(X_{b}^{a^{\prime}} \lambda^{b}\right) / d u=X_{b}^{a^{\prime}}\left(d \lambda^{b} / d u\right)+X_{b c}^{a^{\prime}}\left(d x^{c} / d u\right) \lambda^{b}
\end{equation}

Se $ d \lambda^{a} / d u $ fossem componentes de vetor, o termo $ X_{b c}^{a^{\prime}} \equiv \partial^{2} x^{a^{\prime}} / \partial x^{b} \partial x^{c} $ não existiria. O motivo da presença desse termo está na definição da derivada utilizada:

\begin{equation}\label{eq:DerivadaTotalLambda}
\frac{d \lambda^{a}}{d u} \equiv \lim _{\delta u \rightarrow 0} \frac{\lambda^{a}(u+\delta u)-\lambda^{a}(u)}{\delta u}
\end{equation}

Aqui, tiramos a diferença de componentes em \textit{pontos diferentes} de $ \gamma $. Uma vez que, em geral, as transformações de coeficientes dependem na posição, temos que $ \left(X_{b}^{a^{\prime}}\right)_{u} \neq\left(X_{b}^{a^{\prime}}\right)_{u+\delta u} $, o que significa que as diferenças de componentes não são componentes de um vetor. No limite, a diferença de $ \left(X_{b}^{a^{\prime}}\right)_{u} \text{e } \left(X_{b}^{a^{\prime}}\right)_{u+\delta u} $ aparece como $ X_{b c}^{a^{\prime}} $. Para resolver esse problema, precisamos tomar a diferença entre as componentes no mesmo ponto de $ \gamma $, e podemos fazer isso com a noção de transporte paralelo introduzida anteriormente.

Seja $ P $ o ponto em $ \gamma $ com parâmetro $ u $ e $ Q $ um ponto na vizinhança com parâmetro $ u+\delta u $. Então, $ \lambda^{a}(u+\delta u) $ é um vetor em $ Q $, assim como o vetor $ \bar{\lambda}^{a} $ obtido pelo transporte paralelo de $ \lambda^{a}(u) $ de $ P $ até $ Q$. Desse modo, a diferença $ \lambda^{a}(u+\delta u)-\bar{\lambda}^{a} $ é um vetor em $ Q $ e, portanto, o quociente $ \left(\lambda^{a}(u+\delta u)-\bar{\lambda}^{a}\right) / \delta u $ também o é. É o limite desse quociente (ao passo que $ \delta u \rightarrow 0 $) que chamamos da \textit{derivada absoluta} $ D \lambda^{a} / d u $ de $ \lambda^{a}(u) $ ao longo de $ \gamma $. Mas

\[
\lambda^{a}(u+\delta u) \approx \lambda^{a}(u)+\frac{d \lambda^{a}}{d u} \delta u
\]

E, da equação \textbf{eq:AproximacaoTransporteParalelo}, tiramos

\[
\bar{\lambda}^{a} \approx \lambda^{a}(u)-\Gamma_{b c}^{a} \lambda^{b}(u) \delta x^{c}
\]

Assim,

\[
\frac{\lambda^{a}(u+\delta u)-\bar{\lambda}^{a}}{\delta u} \approx \frac{d \lambda^{a}}{d u}+\Gamma_{b c}^{a} \lambda^{b}(u) \frac{\delta x^{c}}{\delta u}
\]

Quando $ \delta u \rightarrow 0 $, o ponto $ Q $ tende ao ponto $ P $ e o limite do quociente é 

\begin{equation}\label{eq:DerivadaAbsolutaDefinicao}
\boxed{
\frac{D \lambda^{a}}{d u} \equiv \frac{d \lambda^{a}}{d u}+\Gamma_{b c}^{a} \lambda^{b} \frac{d x^{c}}{d u}
}
\end{equation}

Onde todas as quantidades são avaliadas no mesmo ponto $ P $ em $ \gamma $. Assim, a derivada absoluta de um campo vetorial ao longo de uma curva não depende apenas de sua derivada total (que não gera um campo vetorial), mas dos coeficientes de conexão $ \Gamma_{b c}^{a} $ 

{\color{red} mostrar que a derivada absoluta é campo vetorial (a partir da eq. 2.43)}

\section{*Coordenadas geodésicas}\label{sec:CoordenadasGeodesicas}


\section{*O espaço-tempo da Relatividade Geral}\label{sec:EspacotempoRelatividadeGeral}

O espaço-tempo da Relatividade Restrita é uma variedade pseudo-Riemanniana quadridimensional com um sistema global de coordenadas no qual o tensor de métrica é da forma

\[
\left[\eta_{\mu \nu}\right] \equiv\left[\begin{array}{rrrr}{1} & {0} & {0} & {0} \\ {0} & {-1} & {0} & {0} \\ {0} & {0} & {-1} & {0} \\ {0} & {0} & {0} & {-1}\end{array}\right]
\]


Esses sistemas de coordenadas são chamados de Cartesianos. Eles estão relacionados com as coordenadas mais familiares $ t, x, y, z $ por $ x^0 \equiv ct, x^1\equiv x, x^2\equiv y, x^3 \equiv z $, onde $ c $ é a velocidade da luz.

Um dos requerimentos do espaço-tempo da Relatividade Geral é que, localmente, ele deve ser próximo do espaço-tempo da Relatividade Restrita. {\color{red} Como explicado na seção \ref{sec:CoordenadasGeodesicas}}, podemos construir um sistema de coordenadas em torno de qualquer ponto $ P $ no espaço-tempo da Relatividade Geral no qual $ \left(\Gamma_{\nu \sigma}^{\mu}\right)_{P}=0 $ e $ \left(x^{\mu}\right)_{P}=(0,0,0,0) $. Isso significa que $ \left(\partial_{\sigma} g_{\mu \nu}\right)_{P}=0 $. Desse modo, para pontos próximos de $ P $ onde as coordenadas ($ x^\mu $) são pequenas, o teorema de Taylor nos diz

\begin{equation}\label{eq:TensorMetricaAproximado}
g_{\mu \nu} \approx \eta_{\mu \nu}+\tfrac{1}{2}\left(\partial_{\alpha} \partial_{\beta} g_{\mu \nu}\right)_{P} x^{\alpha} x^{\beta}
\end{equation}



\section{Leis de Newton}\label{sec:LeisDeNewton}

A primeira lei de Newton diz que ``Todo corpo continua seu estado de repouso ou de movimento uniforme em uma linha reta, a menos que seja forçado a mudar aquele estado por forças aplicadas sobre ele''. De fato, em um referencial inercial onde podem-se desconsiderar os coeficientes de conexão $ \Gamma^{\mu}_{\nu\sigma} $, a equação geodésica se reduz a $ d^{2} x^{\mu} / d \tau^{2}=0 $. Para velocidades não relativísticas, temos que $ d\tau/dt \approx 1$, o que resulta em $ d^{2} x^{i} / d t^{2}=0\;(i=1,2,3) $, que é a equação de movimento newtoniana de uma partícula livre de forças.

A segunda lei de Newton é comumente escrita na equação 3-vetorial:

\[\dfrac{d\mathbf{p}}{dt}=\mathbf{F}\]

onde $ \mathbf{p} $ é o momento linear e $ \mathbf{F} $ é a força aplicada. Sua generalização para a Relatividade Geral é dada pela equação \textbf{eq:SegundaLeiDeNewtonGeneralizada}.

Já a terceira lei deve ser tratada com cuidado. Ela diz que ``A toda ação há sempre uma reação oposta e de igual intensidade: as ações mútuas de dois corpos um sobre o outro são sempre iguais e dirigidas em sentidos opostos.''. Para a Relatividade Geral, a ideia de força gravitacional é substituída pela ideia de que um corpo massivo curva o espaço-tempo a seu redor. É importante notar que essa abordagem ignora a curvatura produzida pela partícula seguindo a geodésica, ou seja, ela é uma partícula teste e não se considera seus efeitos sobre o corpo produzindo o campo gravitacional.


\section{*Potencial gravitacional e a geodésica}\label{sec:PotencialGravitacionalGeodesica}

Suponha um sistema de coordenadas no qual o tensor de métrica é dado por

\begin{equation}\label{eq:TensorMetricaLocal}
g_{\mu \nu} \equiv \eta_{\mu \nu}+h_{\mu \nu}
\end{equation}

onde $ h_{\mu \nu} $ é pequeno mas não desprezível. Além disso, considere que o campo gravitacional, expresso por $ h_{\mu\nu} $ é quasi-estático, isto é, $ \partial_{0} h_{\mu \nu} \equiv c^{-1} \partial h_{\mu \nu} / \partial t \ll \partial_{i} h_{\mu \nu} (i=1,2,3) $.

Se, em vez de utilizarmos o tempo próprio $ \tau $ como parâmetro, mas a coordenada do tempo $ t $, definida por {\color{blue}$ x^0 \equiv ct $} %no livro está x_0
, então a equação de geodésica que dá a trajetória de uma partícula livre é da forma

\begin{equation}\label{eq:TrajetoriaEmT}
\frac{d^{2} x^{\mu}}{d t^{2}}+\Gamma_{\nu \sigma}^{\mu} \frac{d x^{\nu}}{d t} \frac{d x^{\sigma}}{d t}=h(t) \frac{d x^{\mu}}{d t}
\end{equation}

Onde

\begin{equation}\label{eq:HDefinicao}
h(t) \equiv-\frac{d^{2} t}{d \tau^{2}}\left(\frac{d t}{d \tau}\right)^{-2}=\frac{d^{2} \tau}{d t^{2}}\left(\frac{d \tau}{d t}\right)^{-1}
\end{equation}

{\color{red} prova no caderno de exercícios, pg. 4 [trocar $ s $ por $ \tau $]}

Dividindo-se por $ c^2 $, a parte espacial da equação \eqref{eq:TrajetoriaEmT} pode ser escrita como

\begin{equation}\label{eq:TrajetoriaParteEspacial}
\frac{1}{c^{2}} \frac{d^{2} x^{i}}{d t^{2}}+\Gamma_{00}^{i}+2 \Gamma_{0 j}^{i}\left(\frac{1}{c} \frac{d x^{j}}{d t}\right)+\Gamma_{j k}^{i}\left(\frac{1}{c} \frac{d x^{j}}{d t}\right)\left(\frac{1}{c} \frac{d x^{k}}{d t}\right)=\frac{1}{c} h(t)\left(\frac{1}{c} \frac{d x^{i}}{d t}\right).
\end{equation}

O último termo da esquerda pode ser ignorado, por estarmos tratando velocidades baixas.

Se definirmos $ h^{\mu \nu} \equiv \eta^{\mu \sigma} \eta^{\nu \rho} h_{\sigma \rho} $, então temos que, para aproximações de primeira ordem em $ h_{\mu \nu} \text { e } h^{\mu \nu} $,

\begin{equation}\label{eq:HCovariante}
g^{\mu \nu}=\eta^{\mu \nu}-h^{\mu \nu} \qq{e} \Gamma_{\nu \sigma}^{\mu}=\tfrac{1}{2} \eta^{\mu \rho}\left(\partial_{\nu} h_{\sigma \rho}+\partial_{\sigma} h_{\nu \rho}-\partial_{\rho} h_{\nu \sigma}\right)
\end{equation}

\begin{proof}
	Suponha que $ g^{\mu \nu}=\eta^{\mu \nu}+f^{\mu \nu} $, onde $ f^{\mu\nu} $ é pequeno. Então,
	 \[\delta_{\nu}^{\mu}=g^{\mu \sigma} g_{\sigma \nu}=\left(\eta^{\mu \sigma}+f^{\mu \sigma}\right)\left(\eta_{\sigma \nu}+h_{\sigma \nu}\right) \approx
	 \eta^{\mu\sigma}\eta_{\sigma\nu}+\eta^{\mu\sigma}h_{\sigma\nu}+f^{\mu\sigma}\eta_{\sigma \nu} \]
	 
	 \[
	 \Rightarrow f^{\mu\sigma}\eta_{\sigma \nu}= (\delta_{\nu}^{\mu} - \eta^{\mu\sigma}\eta_{\sigma\nu})-\eta^{\mu\sigma}h_{\sigma\nu} = -\eta^{\mu\sigma}h_{\sigma\nu}
	 \]
	 
	 Contraindo-se com $ \eta^{\nu \rho} $, chegamos em 
	 
	 \[f^{\mu\rho}=-h^{\mu\rho} \Rightarrow g^{\mu \nu}=\eta^{\mu \nu}-h^{\mu \nu} \]
	 
	 Note, também, que abaixando-se o primeiro índice dos coeficientes de conexão, chegamos em $ \Gamma_{\rho \nu \sigma}=g_{\rho\mu}\Gamma_{\nu \sigma}^\mu=
	 \tfrac{1}{2}g_{\rho\mu}g^{\mu\kappa}(\partial_{\nu}g_{\sigma\kappa}+\partial_{\sigma}g_{\nu\kappa}-\partial_{\kappa}g_{\mu\nu})=
	 \tfrac{1}{2}\left(\partial_{\nu} g_{\sigma \rho}+\partial_{\sigma} g_{\nu \rho}-\partial_{\rho} g_{\nu \sigma}\right)=
	 \tfrac{1}{2}\left(\partial_{\nu} h_{\sigma \rho}+\partial_{\sigma} h_{\nu \rho}-\partial_{\rho} h_{\nu \sigma}\right)$. Contraindo-se com $  g^{\mu \nu}=\eta^{\mu \nu}-h^{\mu \nu} $, temos 
	 
	 \[
	 \Gamma^{\mu}_{\nu\sigma}=
	 g^{\mu\rho}\Gamma_{\rho\nu\sigma}= \tfrac{1}{2}(\eta^{\mu\rho}-h^{\mu\rho})(\partial_{\nu} h_{\sigma \rho}+\partial_{\sigma} h_{\nu \rho}-\partial_{\rho} h_{\nu \sigma})
	 \approx \tfrac{1}{2} \eta^{\mu \rho}\left(\partial_{\nu} h_{\sigma \rho}+\partial_{\sigma} h_{\nu \rho}-\partial_{\rho} h_{\nu \sigma}\right)
	 \]
	 
	 
\end{proof}

Assim, desconsiderando-se os valores $ \partial_{0} h_{\mu \nu} $ ao somá-los com $ \partial_{i} h_{\mu \nu} $,

\[
\Gamma_{00}^{i}=\tfrac{1}{2} \eta^{i \rho}\left(\partial_{0} h_{0 \rho}+\partial_{0} h_{0 \rho}-\partial_{\rho} h_{00}\right) = 
\tfrac{1}{2} \eta^{i \rho}\left(\partial_{0} h_{0 \rho}+\partial_{0} h_{0 }-\partial_{\rho} h_{00}\right)
\]
\[
\approx -\tfrac{1}{2} \eta^{i j} \partial_{j} h_{00}=\tfrac{1}{2} \delta^{i j} \partial_{j} h_{00}
\]

e

\[
\Gamma_{0 j}^{i}=\tfrac{1}{2} \eta^{i \rho}\left(\partial_{0} h_{j \rho}+\partial_{j} h_{0 \rho}-\partial_{\rho} h_{0 j}\right)
\]
\[
\approx -\tfrac{1}{2} \delta^{i k}\left(\partial_{j} h_{0 k}-\partial_{k} h_{0 j}\right)
\]

Assim, aproximamos todos os termos do lado esquerdo da equação \eqref{eq:TrajetoriaParteEspacial}. Analogamente, negligenciando produtos de $ c^{-1} d x^{i} / d t $, temos que, a partir de

\[
\left(\frac{d \tau}{d t}\right)^{2}=\frac{1}{c^{2}} g_{\mu \nu} \frac{d x^{\mu}}{d t} \frac{d x^{\nu}}{d t}
\]

que

\[
d \tau / d t\approx \left(g_{00}\left({d(ct)}/dt\right)^2c^{-2}\right)^{1/2} = \left(1+h_{00}\right)^{1 / 2}\approx 1+\tfrac{1}{2} h_{00}
\]

Assim,

\[
d^{2} \tau / d t^{2}=\tfrac{1}{2} c h_{00,0} \Rightarrow \frac{1}{c} h(t)=\tfrac{1}{2} h_{00,0}\left(1-\tfrac{1}{2} h_{00}\right)=\tfrac{1}{2} h_{00,0},
\]

a partir da equação \eqref{eq:HDefinicao}. 

Assim, o lado direito da equação \eqref{eq:TrajetoriaParteEspacial} pode ser desprezado e, então, a aproximação em primeira ordem nos dá

\[
\frac{1}{c^{2}} \frac{d^{2} x^{i}}{d t^{2}}+\tfrac{1}{2} \delta^{i j} \partial_{j} h_{00}-\delta^{i k}\left(\partial_{j} h_{0 k}-\partial_{k} h_{0 j}\right) \frac{1}{c} \frac{d x^{j}}{d t}=0
\]

Introduzindo a massa $ m $ da partícula e rearranjando os termos, temos

\begin{equation}\label{eq:TrajetoriaParticulaComMassa}
m \frac{d^{2} x^{i}}{d t^{2}}=-m \delta^{i j} \partial_{j}\left(\tfrac{1}{2} c^{2} h_{00}\right)+m c \delta^{i k}\left(\partial_{j} h_{0 k}-\partial_{k} h_{0 j}\right) \frac{d x^{j}}{d t}
\end{equation}

Analisando esse resultado em termos da segunda lei de Newton, o lado esquerdo da equação é a massa vezes a aceleração da partícula, então o lado direito é a ``força gravitacional'' atuando sobre ela. O primeiro termo é a força $ -m \nabla V $ gerado pelo potencial $ V \equiv \tfrac{1}{2} c^{2} h_{00} $, ao passo que o segundo termo depende da velocidade e indica um termo rotacional. Se definirmos um sistema de coordenadas em que $ \partial_{j} h_{0 k}-\partial_{k} h_{0 j} $ é não-rotacional, temos que, para uma partícula em velocidades baixas em um sistema inercial, não-rotacional, em que vale a condição quasi-estática, vale que

\begin{equation}\label{eq:EquacaoMovimentoPotencial}
d^{2} x^{i} / d t^{2}=-\delta^{i j} \partial_{j} V,
\end{equation}

com

\begin{equation}\label{eq:PotencialGravitacionalNewtonianoFraco}
V \equiv \tfrac{1}{2} c^{2} h_{00}+V_0.
\end{equation}

Essa é a equação de movimento newtoniana de uma partícula se movendo em um potencial gravitacional $ V $ dado por \eqref{eq:PotencialGravitacionalNewtonianoFraco}. Isso nos dá

\[
g_{00}=1+ 2(V-V_0) / c^{2}.
\]

Escolhendo $ V_0=0 $ de modo que $ g_{00} = 1 $ quando $ V=0 $, de modo a retornarmos ao caso plano, temos que

\begin{equation}\label{eq:RelacaoG00Potencial}
\boxed{
g_{00}=1+2 V / c^{2}}
\end{equation}

é a relação entre $ g_{00} $ e o potencial newtoniano $ V $ para essa aproximação.

\section{Lei da gravitação universal de Newton}\label{sec:GravitacaoNewton}

A solução de Schwarzschild é uma solução exata das equações de campo da Relatividade Geral e pode-se interpretá-la como representando o campo produzido por um corpo massivo. Como deduzido no capítulo \ref{cap:EquacoesDeCampoCurvatura}, seu elemento de linha é

\[c^{2} d \tau^{2}=\left(1-2 G M / r c^{2}\right) c^{2} d t^{2}-\left(1-2 G M / r c^{2}\right)^{-1} d r^{2}-r^{2} d \theta^{2}-r^{2} \sin ^{2} \theta d \phi^{2},\]

onde $ M $ é a massa do corpo e $ G $ a constante gravitacional. Para pequenos valores de $ GM/rc^2 $ isso se aproxima do elemento de linha do espaço-tempo plano em coordenadas esféricas, onde $ r $ é a distância radial. Se definirmos as coordenadas

\[x^{0} \equiv c t, \quad x^{1} \equiv r \sin \theta \cos \phi, \quad x^{2} \equiv r \sin \theta \sin \phi, \quad x^{3} \equiv r \cos \theta,\]

obtemos um elemento de linha cujo tensor de métrica tenha a forma $ g_{\mu\nu} =\eta_{\mu \nu}+h_{\mu\nu} $, em que, para valores grandes de $ rc^2/GM $, as quantidades $ h_{\mu\nu} $ são pequenas e $ g_{00}=1-2GM/rc^2 $.  Isso nos dá $ h_{00} = -2GM/rc^2 $ e, de acordo com os resultados da seção \ref{sec:PotencialGravitacionalGeodesica}, um potencial newtoniano $ V = -GM/r $. A versão vetorial da equação \eqref{eq:EquacaoMovimentoPotencial} é 

\[
m d^{2} \mathbf{r} / d t^{2}=-m \nabla V=-G M m r^{-2} \hat{\mathbf{r}}
,\]

onde $ \mathbf{r} \equiv\left(x^{1}, x^{2}, x^{3}\right) $, $ m $ é a massa da partícula de teste e $ \hat{\mathbf{r}}\equiv \mathbf{r}/|\mathbf{r}| $. Esse resultado corrobora com o esperado ao aplicar-se a segunda lei de Newton na lei da gravitação universal, de modo que essa lei é recuperada como uma aproximação válida para grandes valores de $ rc^2/GM $ e partículas com velocidades baixas.



\section{*Sistema de referencial giratório}\label{sec:SistemaReferencialGiratorio}
