\chapter{Relatividade Especial}\label{cap:RelatividadeEspecial}

\section{Postulados da Relatividade Especial}\label{sec:PostuladosRelatividadeEspecial}

A Relatividade Especial baseia-se em dois postulados:

\begin{enumerate}[label=\textbf{(\Roman*)}]
    \item As leis da física se aplicam a todos os referenciais inerciais \label{post:leis}
    \item A velocidade da luz é a mesma para todos os referenciais inerciais. \label{post:velocidade}
\end{enumerate}

O primeiro postulado nos dá uma noção de referenciais preferenciais. Um referencial inercial é um em que um objeto inicialmente em repouso continuará em repouso se a resultante das forças for zero. Por causa da gravidade, referenciais inerciais devem estar em queda livre, mas a Relatividade Especial descreve eventos sem gravidade, de modo que, na prática, podemos descrever referenciais inerciais em termos de seu movimento relativo em velocidade constante.

O postulado \ref{post:leis} é uma generalização do princípio da \textit{Invariância de Galileu}, de modo a abranger a eletrodinâmica, além da mecânica. No entanto, as equações de Maxwell referenciam explicitamente a velocidade da luz. De fato, ela é escrita em termos da permissividade elétrica $\epsilon_0$ e a permeabilidade magnética $\mu_0$ do vácuo, que são constantes que podem ser medidas experimentalmente. As equações de Maxwell prevêm que as ondas eletromagnéticas (incluindo a luz) possuem velocidade (no vácuo) de 
\[
    c = \frac{1}{\sqrt{\epsilon_0\mu_0}}
\]
No entanto, não há uma referência a qual referencial essa velocidade é observada. O experimento de Michelson-Morley, realizado em 1887, foi realizado a fim de mostrar que essa velocidade é em relação ao éter, de modo que nós poderíamos medir nosso próprio movimento em relação ao éter medindo as variações de $c$ em diferentes direções. No entanto, o experimento mostrou que tais variações não existem. Einstein argumentou que não existe, portanto, éter. O postulado \ref{post:leis} junto com as equações de Maxwell então, nos leva ao postulado \ref{post:velocidade}.

Uma conclusão imediata que segue de ambos os postulados é que dois observadores podem não concordar na simultaneidade entre dois eventos -- na verdade, de maneira geral irão discordar. Por exemplo, considere um trem com uma lâmpada no centro (Figura \ref{fig:Trens} \subref{fig:trem}).

Para um observador dentro do trem, independentemente do movimento do trem, os dois feixes de luz chegarão às paredes ao mesmo tempo. No entanto, para um observador que observe o trem movendo-se para a direita, o feixe viajando para a esquerda chegará primeiro (Figura \ref{fig:Trens} \subref{fig:TremMovimento}). Para esse observador, o trem possui uma velocidade adicional $v$, mas a luz continua com a mesma velocidade $c$ (ambos os feixes), segundo o postulado \ref{post:velocidade}, resultando em distâncias diferentes para serem percorridas até as paredes.


\begin{figure*}[t]
    \centering
    \begin{subfigure}[t]{0.49\textwidth}
        \centering
        \plot{0.83\linewidth}{figuras/trem}
        \caption{}
        \label{fig:trem}
    \end{subfigure}%
    ~ 
    \begin{subfigure}[t]{0.49\textwidth}
        \centering
        \plot{0.9\linewidth}{figuras/TremMovimento}
        \caption{}
        \label{fig:TremMovimento}
    \end{subfigure}
    \caption{ {\small Feixes de luz partindo do centro de um trem para (a) um observador que anda junto ao trem e (b) para um observador que vê o trem movendo-se com velocidade $v$ para a direita.}  }
\label{fig:Trens}
\end{figure*}

\section{Dilatação temporal e contração espacial}\label{sec:DilatacaoContracao}
Vimos um exemplo em que os postulados da relatividade chegam à conclusão de que a passagem do tempo depende do observador. Agora, estudaremos esse fenômeno com mais detalhe.

Considere um trem com altura $h$ e um raio de luz refletindo de cima para baixo entre espelhos posicionados no chão e no teto. O tempo entre os ricocheteios pode ser utilizado como unidade temporal e, \textit{ao ser medido por um observador no trem}, não depende do fato de o trem estar se movendo ou não. Podemos entender esse sistema como um relógio. 

No entanto, para um observador no chão, o feixe de luz aparenta mover-se na diagonal -- de modo que ele sempre acompanhe os espelhos, que se movem -- e, portanto, viajam por um caminho maior. Entretanto, segundo o postulado \ref{post:velocidade}, a luz deve possuir a mesma velocidade para ambos os referenciais. Assim, há uma diferença na taxa em que o tempo é medido pelo 'relógio'.

Seja $t$ o tempo medido por um observador no chão e $t^\prime$ o tempo medido por um observador no trem (similarmente, denotaremos $x, x^\prime$ para medidas de comprimento). Um ricocheteio dos espelhos, medido no trem, leva um tempo $\Delta t^\prime$, de modo que a distância percorrida é $h = c\Delta t^\prime$.

Agora suponha que o mesmo feixe de luz leve um tempo $\Delta t$, medido no chão, para realizar sua trajetória. Nesse tempo, a luz viaja uma distância $c\Delta t$, que é a hipotenusa de um triângulo retângulo com catetos $h$ e $v\Delta t$, como mostrado na Figura \ref{fig:Pitagoras}.
\begin{figure}[ht]
    \centering
    \plot{0.2\linewidth}{figuras/Pitagoras}
    \caption{Triângulo retângulo resultante da comparação entre as trajetórias percorridas pela luz em ambos os referenciais.}
    \label{fig:Pitagoras}
\end{figure}

O teorema de pitágoras nos leva a 
\[
\left(c\Delta t\right)^2 = \left(v\Delta t\right)^2 + \left( c \Delta t^\prime \right) ^2 .
\]
Resolvendo para $\Delta t$, obtemos a relação
\begin{equation}\label{eq:DilatacaoTemporal}
    \Delta t = \gamma\Delta t^\prime, \qq{onde} \gamma=\frac{1}{\sqrt{1-v^2/c^2}}
\end{equation}

Esse efeito é chamado de \textit{dilatação temporal}.

No entanto, o tempo não é a única grandeza sobre a qual dois observadores podem discordar. Considere um raio de luz viajando entre as extremidades de um trem. Se um raio demora um tempo $\Delta t^\prime$ para ir e voltar e o tamanho do trem é $\Delta x^\prime$, então temos que 
\begin{equation}\label{eq:RelacaoXlinhaTlinha}
    c\Delta t^\prime = 2\Delta x^\prime .
\end{equation}


Agora, o que um observador no chão vê? Como o trem está em movimento, a distância percorrida em uma direção é diferente da outra, isto é, a luz que começa na parte de trás do trem precisa alcançar a frente, enquanto a parte de trás vai de encontro com um feixe indo no sentido contrário. Se um tempo $\Delta t_1$ passa para que se alcance a parte da frente, então a distância percorrida é $\Delta x + v\Delta t_1$, somando-se o tamanho do trem (de acordo com o referencial no chão) e a distância que o trem percorreu enquanto a luz viajava. Para o sentido contrário, num tempo $\Delta t_2$, a distância percorrida é $\Delta x-v\Delta t_2$. Assim, temos 
\begin{align*}
    &c\Delta t_1 = \Delta x + v\Delta t_1 \Rightarrow \Delta t_1 = \frac{\Delta x}{c-v}\\
    &c\Delta t_2 = \Delta x - v\Delta t_2 \Rightarrow \Delta t_2 = \frac{\Delta x}{c+v}
\end{align*}
Combinando ambos, chegamos em
\[
    c\Delta t = c\Delta t_1 + c\Delta t_2 = \frac{2\Delta x}{1-v^2/c^2}
\]
combinando isso com as equações \eqref{eq:DilatacaoTemporal} e \eqref{eq:RelacaoXlinhaTlinha}, chegamos em
\begin{equation}\label{eq:ContracaoEspacial}
    \Delta x = \sqrt{1-\frac{v^2}{c^2}}\Delta x^\prime = \frac{1}{\gamma}\Delta x^\prime
\end{equation}
Assim, um objeto em movimento aparenta ser menor na direção de movimento. Esse efeito é conhecido como \textit{contração espacial}.

\section{Transformações de Lorentz}\label{sec:TransformacoesLorentz}

Suponha que um observador $O^\prime$ se move para a direita com velocidade $v$. Se $x$ denota a distância de um objeto a um observador parado $O$, então a distância entre o objeto e $O^\prime$, medido por $O$, será $x-vt$. Mas, utilizando $\Delta x^\prime = \gamma \Delta x$, temos que
\begin{equation}\label{eq:LorentzX}
    x^\prime = \gamma (x-vt).
\end{equation}
Pelo postulado \ref{post:leis}, se trocarmos o papel de $O$ e $O^\prime$, nada deve mudar, exceto o fato de que a velocidade relativa agora é $-v$. Por simetria, pode-se concluir então que
\begin{equation}\label{eq:LorentzXInversa}
    x = \gamma (x^\prime + vt^\prime) .
\end{equation}
Combinando ambas as equações, obtemos
\begin{equation}\label{eq:LorentzT}
    t^\prime = \gamma \left(t - \frac{v}{c^2}x\right)
\end{equation}
e, pelo mesmo argumento utilizado anteriormente, podemos invertê-la como
\begin{equation}\label{eq:LorentzTInversa}
    t = \gamma \left(t^\prime + \frac{v}{c^2}x^\prime\right) .
\end{equation}

As equações \eqref{eq:LorentzX} e \eqref{eq:LorentzT} são as transformações de Lorentz de $O$ para $O^\prime$ e, trocando-se $v$ por $-v$, obtemos as inversas \eqref{eq:LorentzXInversa} e \eqref{eq:LorentzTInversa}.

Utilizando as transformações de Lorentz, vemos que
\begin{equation}\label{eq:Intervalo}
\begin{aligned} c^{2} t^{\prime 2} -x^{\prime2} &=\gamma^{2}\left(c t-\frac{v}{c} x\right)^{2}-\gamma^{2}(x-v t)^{2} \\ &=c^{2} t^{2}-x^{2} ,\end{aligned}
\end{equation}
de modo que a quantidade $c^2t^2-x^2$, conhecida como o \textit{intervalo de espaço-tempo}, não depende do observador que o computa.

\section{Geometria Hiperbólica}\label{sec:GeometriaHiperbolica}

As funções hiperbólicas podem ser definidas na seguinte maneira:

\begin{equation}\label{eq:FuncoesHiperbolicas}
 \cosh \beta =\frac{e^{\beta}+e^{-\beta}}{2}, \quad \sinh \beta =\frac{e^{\beta}-e^{-\beta}}{2}, \quad \tanh \beta =\frac{\sinh \beta}{\cosh \beta } \; .
\end{equation}

É direto verificar que as seguintes identidades são válidas:
\begin{equation}\label{eq:IdentidadesFuncoesHiperbolicas}
\arraycolsep=1.4pt\def\arraystretch{2.2}
\begin{array}{lr}
{\sinh (\alpha+\beta)=\sinh \alpha \cosh \beta+\cosh \alpha \sinh \beta \;} & {\frac{d}{d \beta} \sinh \beta=\cosh \beta} \\
{\cosh (\alpha+\beta)=\cosh \alpha \cosh \beta+\sinh \alpha \sinh \beta \;} & {\frac{d}{d \beta} \cosh \beta=\sinh \beta} \\
{\tanh (\alpha+\beta)=\dfrac{\tanh \alpha+\tanh \beta}{1+\tanh \alpha \tanh \beta}} & {\cosh ^{2} \beta-\sinh ^{2} \beta=1  }
\end{array}
\end{equation}

A geometria euclidiana é baseada em círculos. Isto é, os pontos cujas distâncias à origem são uma constante (por exemplo, $\rho$) formam um conjunto de círculos (neste caso, a circunferencia $\rho^2=x^2+y^2$, para o plano). Para a geometria hiperbólica, apenas trocamos a função de distância. Um ponto $B$ com coordenadas $(x,y)$ tem ``distância quadrática'' da origem definida por
\begin{equation}\label{eq:DistanciaHiperbolica}
    \rho^2=x^2-y^2 .
\end{equation}

Assim, os ``círculos'' de distância constante à origem se tornam hipérboles de $\rho=\text{constante}$. Restringimos a $x>0$ e definimos o \textit{ângulo hiperbólico} $\beta$ entre a linha da origem até um ponto $B$ da hipérbole e o eixo $x$ como sendo a razão entre o comprimento Lorentziano\footnote{Na geometria euclidiana, o comprimento de uma curva é obtido ao integrar-se $ds$ ao longo da curva, onde $ds^2=dx^2+dy^2$. Analogamente, o comprimento Lorentziano é obtido ao integrar $d\sigma$, onde $d\sigma^2=|dx^2-dy^2|$} do arco da hipérbole entre $B$ e $(\rho,0)$. Em termos das coordenadas $(x,y)$ de $B$, temos que
\begin{equation}\label{eq:AnguloHiperbolico}
    \cosh\beta = \frac{x}{\rho} ,\quad
    \sinh\beta = \frac{y}{\rho} .
\end{equation}

\begin{figure}[t]
    \centering
    \plot{0.36\linewidth}{figuras/Hiperbole}
    \caption{Construção geométrica do ângulo hiperbólico}
    \label{fig:Hiperbole}
\end{figure}

Pode-se mostrar que essa definição é igual a dada pelas equações \eqref{eq:FuncoesHiperbolicas}. Essa construção é mostrada na Figura \ref{fig:Hiperbole}, onde a outra hipérbole é dada por $x^2-y^2=-\rho^2$. Por simetria, o ponto $A$ nessa hipérbole tem coordenadas $(x,y) = (\rho\sinh\beta,\rho\cosh\beta)$.

Por analogia com a rotação euclidiana, definimos a rotação hiperbólica pelas relações
\begin{equation}\label{eq:RotacaoHiperbolica}
\left(\begin{array}{l}{x} \\ {y}\end{array}\right)=\left(\begin{array}{ll}{\cosh \beta} & {\sinh \beta} \\ {\sinh \beta} & {\cosh \beta}\end{array}\right)\left(\begin{array}{l}{x^{\prime}} \\ {y^{\prime}}\end{array}\right) .
\end{equation}
Isso é equivalente a ``rotacionar'' os eixos $x, y$ no primeiro quadrante, como mostrado pela Figura \ref{fig:Hiperbole}, encontrando novas coordenadas $(x^\prime,y^\prime)$. A partir da identidade $\cosh^2\beta-\sinh^2\beta=1$, pode-se verificar que a ``distância'' é invariante, isto é,
\begin{equation}\label{eq:InvarianciaDistanciaHiperbolica}
    x^2-y^2=x^{\prime2}-y^{\prime 2} .
\end{equation}

Pode-se perceber que, ao utilizarmos as coordenadas $(x,ct)$, esta equação é igual à identidade \eqref{eq:Intervalo}. Além disso, para essas coordenadas, uma reta de inclinação $\beta$, isto é, uma reta com $\Delta x/(c\Delta t)=\tanh\beta$ simboliza a linha de tempo de um referencial com velocidade relativa $v = \Delta x/\Delta t$, de modo que $\tanh\beta=v/c$.

Em vista disso, podemos calcular o valor de $\gamma$ para essa mudança de referencial, utilizando as identidades \eqref{eq:IdentidadesFuncoesHiperbolicas}
\begin{equation}\label{eq:GammaCosh}
    \gamma = \frac{1}{\sqrt{1-v^2/c^2}}=\frac{1}{\sqrt{1-\tanh^2\beta}}
    =\sqrt{\frac{\cosh^2\beta}{\cosh^2\beta-\sinh^2\beta}}=\cosh\beta ,
\end{equation}
além de que $\tfrac{v}{c}\gamma = \tanh\beta\cosh\beta=\sinh\beta$. Agora, se reescrevermos as transformações de Lorentz \eqref{eq:LorentzX},\eqref{eq:LorentzT}, deixando $ct$ em evidência, temos
\begin{equation}
\begin{aligned} x &=\gamma\left(x^{\prime}+\frac{v}{c} c t^{\prime}\right) \\ c t &=\gamma\left(c t^{\prime}+\frac{v}{c} x^{\prime}\right) \end{aligned}
\end{equation}
Mas, utilizando as relações anteriores, chegamos que isso é equivalente a
\begin{equation}
\begin{aligned} x &=x^{\prime} \cosh \beta+c t^{\prime} \sinh \beta \\ c t &=x^{\prime} \sinh \beta+c t^{\prime} \cosh \beta ,\end{aligned}
\end{equation}
cuja forma matricial é a equação \eqref{eq:RotacaoHiperbolica}, com $y=ct$. Assim, podemos ver que a geometria hiperbólica é a geometria da Relatividade Especial.

\section{Mecânica Relativística}
Seja $\tau$ o tempo medido por um relógio carregado por um observador movendo-se a velocidade constante $u$ com respeito a um referencial inicial    dado. Nós chamamos $\tau$ de \textit{tempo próprio}. A partir da invariância do intervalo, temos que
\[
  c^2dt^2-dx^2=c^2d\tau^2 -0 \Rightarrow d\tau^2 =  
  \left( 1-\tfrac{1}{c^2}\left( dx/dt \right)^2 \right)dt^2,
\]
ou, de maneira equivalente, utilizando a relação \eqref{eq:GammaCosh}
\[
    d\tau = \sqrt{1-u^2/c^2} \,dt = \frac{1}{\gamma}dt= \frac{1}{\cosh\alpha}dt.
\]
Esse cálculo dá o mesmo resultado em qualquer referencial; o tempo próprio independe de referencial.

Agora, temos que a quantidade $p=mu=mdx/dt$ \emph{não} é conservada. Usa-se então o momento definido utilizando-se o tempo próprio:
\[
    p = m\frac{dx}{d\tau}=mc\sinh\alpha.
\]
Suponha que, para um referencial inercial específico, o momento total de uma coleção de partículas é o mesmo antes e depois de uma interação, isto é,
\[
   \sum m_i c\sinh\alpha_i = \sum M_jc\sinh A_j .
\]
Agora veja a situação vista por outro referencial inercial, movendo-se em relação ao primeiro com velocidade $v = c\tanh\beta$. Temos então que $\alpha_i = \alpha'_i+\beta$ e $A_j = A'_j+\beta$. Assim,
\[
\begin{aligned} \sum m_{i} c \sinh \alpha_{i}^{\prime} &=\sum m_{i} c \sinh \left(\alpha_{i}-\beta\right) \\ &=\left(\sum m_{i} c \sinh \alpha_{i}\right) \cosh \beta-\left(\sum m_{i} c \cosh \alpha_{i}\right) \sinh \beta \end{aligned}  
\]
e
\[
    \sum M_{j} c \sinh A_{j}^{\prime}=\left(\sum M_{j} c \sinh A_{j}\right) \cosh \beta-\left(\sum M_{j} c \cosh A_{j}\right) \sinh \beta   .
\]
Os coeficientes de $\cosh\beta$ nessas duas expressões são iguais devido a suposição da conservação do momento no referencial original. Assim, a conservação do momento será válida no novo referencial se, e somente se, os coeficientes de $\sinh\beta$ também concordarem, isto é, 
\[ \sum m_i c\cosh\alpha_i = \sum M_jc\cosh A_j\]
e, de fato, definimos isso como a energia da partícula.
Podemos, agora, definir a grandeza vetorial 2-momento $\mathbf{p}$ (pois estamos considerando apenas uma direção espacial, nas seções futuras utilizaremos o análogo 4-momento) como a derivada em relação ao tempo próprio das coordenadas $x, ct$:
\begin{equation}
    \mathbf{p}=m \mqty(cdt/d\tau \\ dx/d\tau) = mc\mqty(\cosh\alpha \\ \sinh\alpha) = \mqty( \tfrac{1}{c}E \\ p).
\end{equation}