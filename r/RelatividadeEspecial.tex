\chapter{Relatividade Especial}

\section{Postulados da Relatividade Especial}

A Relatividade Especial baseia-se em dois postulados:

\begin{enumerate}[label=\textbf{(\Roman*)}]
    \item As leis da física se aplicam a todos os referenciais inerciais \label{post:leis}
    \item A velocidade da luz é a mesma para todos os referenciais inerciais. \label{post:velocidade}
\end{enumerate}

O primeiro postulado nós dá uma noção de referenciais preferenciais. Um referencial inercial é um em que um objeto inicialmente em repouso continuará em repouso. Por causa da gravidade, referenciais inerciais devem estar em queda livre, mas a Relatividade Especial descreve eventos sem gravidade, de modo que, na prática, podemos descrever referenciais inerciais em termos de seu movimento relativo em velocidade constante.

O postulado \ref{post:leis} é uma generalização do princípio da \textit{Invariância de Galileu}, de modo a abranger a eletrodinâmica, além da mecânica. No entanto, as equações de Maxwell referenciam explicitamente a velocidade da luz. De fato, ela é escrita em termos da permissividade elétrica $\epsilon_0$ e a permeabilidade magnética $\mu_0$ do vácuo, que são constantes que podem ser medidas experimentalmente. As equações de Maxwell prevêm que as ondas eletromagnéticas (incluindo a luz) possuem velocidade (no vácuo) de 
\[
    c = \frac{1}{\sqrt{\epsilon_0\mu_0}}
\]
No entanto, não há uma referência a qual referencial essa velocidade é observada. O experimento de Michelson-Morley, realizado em 1887, foi realizado a fim de mostrar que essa velocidade é em relação ao éter, de modo que nós poderíamos medir nosso próprio movimento em relação ao éter ao medirem-se as variações de $c$ dependentes da direção. No entanto, o experimento mostrou que não há tais variações. Einstein então argumentou que não existe, portanto, éter. O postulado \ref{post:leis} junto com as equações de Maxwell então, nos leva ao postulado \ref{post:velocidade}.

Uma conclusão imediata que segue de ambos os postulados é que dois observadores podem não concordar na simultaneidade entre dois eventos -- na verdade, eles discordam, no geral. Por exemplo, considere um trem com uma lâmpada no centro (Figura \ref{fig:Trens} \subref{fig:trem}).

Para um observador dentro do trem, independentemente do movimento deste, os dois feixes de luz chegarão às paredes ao mesmo tempo. No entanto, para um observador que observe o trem movendo-se para a direita, o feixe viajando para a equerda chegará primeiro (Figura \ref{fig:Trens} \subref{fig:TremMovimento}). Para esse observador, o trem possui uma velocidade adicional $v$, mas a luz continua com a mesma velocidade $c$ (ambos os feixes), segundo o postulado \ref{post:velocidade}, resultando em distâncias diferentes para serem percorridas até as paredes.


\begin{figure*}[t]
    \centering
    \begin{subfigure}[t]{0.5\textwidth}
        \centering
        \plot{0.83\linewidth}{figuras/trem}
        \caption{}
        \label{fig:trem}
    \end{subfigure}%
    ~ 
    \begin{subfigure}[t]{0.5\textwidth}
        \centering
        \plot{0.9\linewidth}{figuras/TremMovimento}
        \caption{}
        \label{fig:TremMovimento}
    \end{subfigure}
    \caption{Feixes de luz partindo do centro de um trem ao mesmo tempo para (a) um observador que anda junto ao trem e (b) para um observador que vê o trem movendo-se com velocidade $v$ para a direita.}
\label{fig:Trens}
\end{figure*}

\section{Dilatação temporal e contração espacial}
Vimos um exemplo em que os postulados da relatividade chegam à conclusão de que a passagem do tempo depende do observador. Agora, estudaremos esse fenômeno com mais detalhe.

Considere, agora, um trem com altura $h$ e um raio de luz refletindo de cima a baixo entre espelhos posicionados no chão e no teto. O tempo entre os ricocheteios pode ser utilizado como unidade temporal e, \textit{ao ser medido por um observador no trem}, não depende da fato de o trem estar se movendo ou não. Podemos entender esse sistema como um relógio. 

No entanto, para um observador no chão, o feixe de luz aparenta mover-se na diagonal -- de modo que ele sempre acompanhe os espelhos, que se movem -- e, portanto, viajam por um caminho maior. No entanto, segundo o postulado \ref{post:velocidade}, a luz deve possuir mesma velocidade para ambos os referenciais. Assim, há uma diferença na taxa em que o tempo é medido pelo 'relógio'.

Seja $t$ o tempo medido por um observador no chão e $t^\prime$ o tempo medido por um observador no trem (similarmente, denotaremos $x, x^\prime$ para medidas de comprimento). Um ricocheteio dos espelhos, medido pelo trem, leva um tempo $\Delta t^\prime$, de modo que a distância percorrida é $h = c\Delta t^\prime$.

Agora suponha que o mesmo feixe de luz leve um tempo $\Delta t$, medido no chão, para realizar sua trajetória. Nesse tempo, a luz viaja uma distância $c\Delta t$, que é a hipotenusa de um triângulo retângulo com catetos $h$ e $v\Delta t$, como mostrado na Figura \ref{fig:Pitagoras}.
\begin{figure}[ht]
    \centering
    \plot{0.2\linewidth}{figuras/Pitagoras}
    \caption{Triângulo retângulo resultante da comparação entre as trajetórias percorridas pela luz em ambos os referenciais.}
    \label{fig:Pitagoras}
\end{figure}

O teorema de pitágoras nos leva a 
\[
\left(c\Delta t\right)^2 = \left(v\Delta t\right)^2 + \left( c \Delta t^\prime \right) ^2 .
\]
Resolvendo para $\Delta t$, obtemos a relação
\begin{equation}\label{eq:DilatacaoTemporal}
    \Delta t = \gamma\Delta t^\prime, \qq{onde} \gamma=\frac{1}{\sqrt{1-v^2/c^2}}
\end{equation}

Esse efeito é chamado de \textit{dilatação temporal}.

No entanto, o tempo não é a única coisa em que dois observadores podem discordar. Considere um raio de luz viajando entre as extremidades de um trem. Se um raio demora um tempo $\Delta t^\prime$ para ir e voltar e o tamanho do trem é $\Delta x^\prime$, então temos que 
\begin{equation}\label{eq:RelacaoXlinhaTlinha}
    c\Delta t^\prime = 2\Delta x^\prime .
\end{equation}


Agora, o que um observador no chão vê? Como o trem está em movimento, a distância percorrida em uma direção é diferente da outra, isto é, a luz que começa na parte de trás do trem precisa alcançar a frente, enquanto a parte de trás vai de encontro com um feixe indo no sentido contrário. Se um tempo $\Delta t_1$ passa para que se alcance a parte da frente, então a distância percorrida é $\Delta x + v\Delta t_1$, somando-se o tamanho do trem (de acordo com o referencial no chão) e a distância que o trem percorreu enquanto a luz viajava. Para o sentido contrário, num tempo $\Delta t_2$, a distância percorrida é $\Delta x-v\Delta t_2$. Assim, temos 
\begin{align*}
    &c\Delta t_1 = \Delta x + v\Delta t_1 \Rightarrow \Delta t_1 = \frac{\Delta x}{c-v}\\
    &c\Delta t_2 = \Delta x - v\Delta t_2 \Rightarrow \Delta t_2 = \frac{\Delta x}{c+v}
\end{align*}
Combinando ambos, chegamos em
\[
    c\Delta t = c\Delta t_1 + c\Delta t_2 = \frac{2\Delta x}{1-v^2/c^2}
\]
combinando isso com as equações \eqref{eq:DilatacaoTemporal} e \eqref{eq:RelacaoXlinhaTlinha}, chegamos em
\begin{equation}\label{eq:ContracaoEspacial}
    \Delta x = \sqrt{1-\frac{v^2}{c^2}}\Delta x^\prime = \frac{1}{\gamma}\Delta x^\prime
\end{equation}
Assim, um objeto em movimento aparenta ser menor na direção de movimento. Esse efeito é conhecido como \textit{contração espacial}.

\section{Transformações de Lorentz}

Suponha que um observador $O^\prime$ se move para a direita com velocidade $v$. Se $x$ denota a distância de um objeto a um observador parado $O$, então a distância entre o objeto e $O^\prime$, medido por $O$, será $x-vt$. Mas, utilizando $\Delta x^\prime = \gamma \Delta x$, temos que
\begin{equation}\label{eq:LorentzX}
    x^\prime = \gamma (x-vt).
\end{equation}
Pelo postulado \ref{post:leis}, se trocarmos o papel de $O$ e $O^\prime$, nada deve mudar, exceto o fato de que a velocidade relativa agora é $-v$. Por simetria, pode-se concluir então que
\begin{equation}\label{eq:LorentzXInversa}
    x = \gamma (x^\prime + vt^\prime) .
\end{equation}
Combinando ambas as equações, obtemos
\begin{equation}\label{eq:LorentzT}
    t^\prime = \gamma \left(t - \frac{v}{c^2}x\right)
\end{equation}
e, pelo mesmo argumento utilizado anteriormente, podemos invertê-la como
\begin{equation}\label{eq:LorentzTInversa}
    t = \gamma \left(t^\prime + \frac{v}{c^2}x^\prime\right) .
\end{equation}

As equações \eqref{eq:LorentzX} e \eqref{eq:LorentzT} são as transformações de Lorentz de $O$ para $O^\prime$ e, trocando-se $v$ por $-v$, obtemos as inversas \eqref{eq:LorentzXInversa} e \eqref{eq:LorentzTInversa}.

% \section{Estrutura causal no espaço-tempo de Minkowski}

% \section{Dinâmica Relativística}